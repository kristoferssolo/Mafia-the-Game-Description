\subsection{Sistēmas lietotāji}

Neautentificēts lietotājs (viesis), i.e., viesis ir jebkurš lietotājs, kas nav
pieteicies vai reģistrējies sistēmā. Šiem lietotājiem ir pieejamas funkcijas,
lai reģistrētos vai pieteiktos sistēmā;

Kad lietotājs ir pieteicies un ir autentificēts, tam ir pieejamas reģistrēta
lietotāja grupas privilēģijas, precīzāk, darbības saistītas ar spēli, profilu
un konta pārvaldi. Tā būs vislielākā grupa pēc lietotāju skaita. Maksas
lietotājiem, precīzāk, reģistrētiem lietotājiem, kuriem piesaistīts aktīvs
abonements, tiek piešķirtas papildus funkcijas - izveidot jaunas virtuālās
istabas, izvēlēties spēles konfigurāciju savās istabās un citas. Maksas
lietotāja grupa ir atvasināta no reģistrēta lietotāja grupas.

Administratoru uzdevumi ietver istabu uzturēšanu un  lietotāju moderēšanu ar
darbībām, kā bloķēšana, spēles istabas un lietotāju stāvokļa izmainīšana, konta
informācijas izmaiņa, lomu uzstādījumu un spēles konfigurācijas rediģēšana.
Lietotājs ``Sistēma'' izpilda noteiktas, ar spēles gaitu saistītas darbības, kas
notiek automātiski un kas nav tiešā veidā citu lietotāju grupu kompetencēs.

Ar lietotājiem saistītās datu plūsmas ir attēlotas sistēmas nultā līmeņa DPD (skat. \ref{fig:dpd-0} att.).


\begin{figure}[htbp]
	\centering
	\includegraphics[width=\linewidth]{./src/img/0tāLīmeņaDPD.png}
	\caption{0. līmeņa DPD}
	\label{fig:dpd-0}
\end{figure}

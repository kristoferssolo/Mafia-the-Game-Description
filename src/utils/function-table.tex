% \newcommand{\moduleFunctionTable}[9]{
% 	\begin{tabularx}{\linewidth}{|X|}
% 		\caption{#1}\label{tab:#2}                                   \\ \hline
% 		\endfirsthead
%
% 		\hline \multicolumn{1}{r}{Turpinājums no iepriekšējās lapas} \\ \hline
% 		\endhead
%
% 		\hline \multicolumn{1}{r}{Turpinājums nākamajā lapā}         \\ \hline
% 		\endfoot
%
% 		\hline
% 		\endlastfoot
%
% 		\textbf{Funkcijas nosaukums}                                 \\ \hline
% 		#3                                                           \\ \hline
% 		\textbf{Funkcijas identifikators}                            \\ \hline
% 		#4                                                           \\ \hline
% 		\textbf{Ievads}                                              \\ \hline
% 		#5                                                           \\ \hline
% 		\textbf{Ievade}                                              \\ \hline
% 		#6                                                           \\ \hline
% 		\textbf{Apstrāde}                                            \\ \hline
% 		#7                                                           \\ \hline
% 		\textbf{Izvade}                                              \\ \hline
% 		#8                                                           \\ \hline
% 		\textbf{Paziņojumi}                                          \\ \hline
% 		#9                                                           \\ \hline
% 	\end{tabularx}
% }


% \newcommand{\moduleFunctionTable}[9]{
% 	\paragraph{#1} \label{tab:#2}
% 	\textbf{Funkcijas nosaukums}: #3
%
% 	\textbf{Funkcijas identifikators}: #4
%
% 	\textbf{Ievads}: #5
%
% 	\textbf{Ievade}:
%
% 	#6
%
% 	\textbf{Apstrāde}: #7
%
% 	\textbf{Izvade}: #8
%
% 	\textbf{Paziņojumi}: #9
% }


\mdfsetup{
    innertopmargin=8pt,
    innerbottommargin=8pt,
    skipbelow=0pt,
    skipabove=0pt,
    everyline=true,
    backgroundcolor=gray!2,
    linecolor=black,
    fontcolor=black,
    roundcorner=0pt,
    splittopskip=25pt,
    secondextra={
        \node[
            overlay,
            fill=white,
            anchor=west,
            inner xsep=10pt
        ] at ([xshift=10pt]O|-P) {Turpinājums};
    },
    middleextra={
        \node[
            overlay,
            fill=white,
            anchor=west,
            inner xsep=10pt
        ] at ([xshift=10pt]O|-P) {Excursus (Cont.)};
    }
    }

\newcommand{\moduleFunctionTableItem}[2]{
    \begin{mdframed}
        \textbf{#1}
    \end{mdframed}

    \begin{mdframed}
        #2
    \end{mdframed}
}

\newcommand{\moduleFunctionTable}[9]{
	\paragraph{#1} \label{tab:#2}
	\moduleFunctionTableItem{Funkcijas nosaukums}{#3}
	\moduleFunctionTableItem{Funkcijas identifikators}{#4}
	\moduleFunctionTableItem{Ievads}{#5}
	\moduleFunctionTableItem{Ievade}{#6}
	\moduleFunctionTableItem{Apstrāde}{#7}
	\moduleFunctionTableItem{Izvade}{#8}
	\moduleFunctionTableItem{Paziņojumi}{#9}
}

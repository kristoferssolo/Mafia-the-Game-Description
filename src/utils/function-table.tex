% \newcommand{\moduleFunctionTable}[9]{
% 	\begin{tabularx}{\linewidth}{|X|}
% 		\caption{#1}\label{tab:#2}                                   \\ \hline
% 		\endfirsthead
%
% 		\hline \multicolumn{1}{r}{Turpinājums no iepriekšējās lapas} \\ \hline
% 		\endhead
%
% 		\hline \multicolumn{1}{r}{Turpinājums nākamajā lapā}         \\ \hline
% 		\endfoot
%
% 		\hline
% 		\endlastfoot
%
% 		\textbf{Funkcijas nosaukums}                                 \\ \hline
% 		#3                                                           \\ \hline
% 		\textbf{Funkcijas identifikators}                            \\ \hline
% 		#4                                                           \\ \hline
% 		\textbf{Ievads}                                              \\ \hline
% 		#5                                                           \\ \hline
% 		\textbf{Ievade}                                              \\ \hline
% 		#6                                                           \\ \hline
% 		\textbf{Apstrāde}                                            \\ \hline
% 		#7                                                           \\ \hline
% 		\textbf{Izvade}                                              \\ \hline
% 		#8                                                           \\ \hline
% 		\textbf{Paziņojumi}                                          \\ \hline
% 		#9                                                           \\ \hline
% 	\end{tabularx}
% }


\newcommand{\moduleFunctionTable}[9]{
	\paragraph{#1} \label{tab:#2}
	\textbf{Funkcijas nosaukums}: #3

	\textbf{Funkcijas identifikators}: #4

	\textbf{Ievads}: #5

	\textbf{Ievade}:

	#6

	\textbf{Apstrāde}: #7

	\textbf{Izvade}: #8

	\textbf{Paziņojumi}: #9
}

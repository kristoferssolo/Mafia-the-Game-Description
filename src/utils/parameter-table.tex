\newcommand{\parameterTable}[7]{
	% \paragraph{#1}
	\begin{table}[h]
		\caption{#1}\label{tab:#2}
	\end{table}
	\specificationTableItem{Parametra nosaukums}{#3}
	\specificationTableItem{Parametra identifikators}{#4}
	\specificationTableItem{Parametra apraksts}{#5}
	\specificationTableItem{Parametra prasības}{#6}
	\specificationTableItem{Parametra piemērs}{#7}
}

% \newcommand{\parameterTable}[7]{
% 	\begin{tabularx}{\linewidth}{|X|}
% 		\caption{#1} \label{tab:#2}                                  \\ \hline
% 		\endfirsthead
% 		\hline \multicolumn{1}{r}{Turpinājums no iepriekšējās lapas} \\ \hline
% 		\endhead
%
% 		\hline \multicolumn{1}{r}{Turpinājums nākamajā lapā}         \\ \hline
% 		\endfoot
%
% 		\hline
% 		\endlastfoot
%
% 		\textbf{Parametra nosaukums}                                 \\ \hline
% 		#3                                                           \\ \hline
% 		\textbf{Parametra identifikators}                            \\ \hline
% 		#4                                                           \\ \hline
% 		\textbf{Parametra apraksts}                                  \\ \hline
% 		#5                                                           \\ \hline
% 		\textbf{Parametra prasības}                                  \\ \hline
% 		#6                                                           \\ \hline
% 		\textbf{Parametra piemērs}                                   \\ \hline
% 		#7                                                           \\ \hline
% 	\end{tabularx}
% }

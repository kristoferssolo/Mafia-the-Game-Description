\subsection*{Darbības sfēra}

Sistēma ``Mafija'' ir atvasināta no plaši pazīstamas sociālas lomu spēles, kas
balstās dedukcijā. Spēlē piedalās indivīdi - Spēlētāji, kas sadalīti vairākās
grupās un tajās ietvertās lomās. Lomu grupa ``Ciems'' lomas ``Iedzīvotājs''
ietvaros cenšas izdibināt kuri ir lomu grupas ``Mafija'' locekļi. Mafijas mērķis
ir radīt haosu ciema iedzīvotāju vidū un pakāpeniski izslēgt ciema iedzīvotājus
no spēles, izmantojot stratēģisku manipulāciju vai iedalītās lomas darbības.
Spēlētāji, kuri nav ietverti ne ``Ciems'', ne ``Mafija'' lomu grupā cenšas sasniegt
tiem iedalītās lomas mērķi. Tikai ``Mafijas'' locekļiem ir informācija par to,
kuri no spēlētāju loka pieder ``Mafija'' lomu grupai. Katram spēlētājam jāizmanto
individuāla ierīce, kas var pieslēgties tīmeklim, lai pieteiktos sistēmā,
pievienotos konkrētajai spēlei un piedalītos tajā.

Katra spēlētāja ierīcē spēles sesijas laikā tiek parādīta informācija par
iedalīto lomu un ar to saistītajām, pieejamajām darbībām, kuru nav paredzēts
vai atļauts rādīt citiem spēlētājiem. Sistēmas vizuālā saskarne ietver
informāciju par spēles aktuālo stāvokli, precīzāk, fāzi (diena / nakts), spēles
ilgumu, palikušo spēlētāju skaitu un citiem spēli raksturojošiem faktoriem.

Spēlētāja darbību klāsts ir atkarīgas no iedalītās lomas un aktuālā spēles
stāvokļa. Spēles organizatoram (maksas lietotājam) ir iespēja izveidot virtuālu
telpu un pielāgot tās iestatījumus, lai organizētu spēli vai mainītu to
konfigurāciju, kas ietver noteiktās lomas, kā arī mainīt un veidot jaunas
lomas.

Katram spēlētājam tiek nodrošināta sinhronizēta informācija par spēles tekošo
stāvokli un pieejamajām darbībām, tai skaitā, paziņojumi par spēles stāvokļa
izmaiņām.

Ārpus spēles sesijas, lietotājiem ir pieejams spēļu istabu saraksts, kas var
ietvert gan atvērtas, gan privātas virtuālās spēļu telpas, statistikas
pārskats, kurā pieejama statistika par jau izspēlētajām spēlēm, un lietotāja
profils, kurā var rediģēt lietotāju raksturojošo informāciju.

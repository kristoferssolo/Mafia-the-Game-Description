\subsection*{Pārskats}

Dokumenta ievads satur tā nolūku, izstrādājamās programmatūras skaidrojumu,
vispārīgu programmatūras mērķi un funkciju klāstu, saistību ar citiem
dokumentiem, kuru prasības tika izmantotas dokumenta izstrādāšanas gaitā, kā
arīpārskatu par dokumenta daļu saturu ar dokumenta struktūras skaidrojumu.

Pirmajā nodaļa tiek aprakstīti faktori, kas var ietekmēt produktu un tā
prasības. Nodaļā tiek pamatota programmatūras izstrādes motivācija un nolūks,
aprakstītas produkta vieta citu sistēmu perspektīvā, galvenās augsta līmeņa
darījumprasības, sistēmas lietotāju grupu lomas un mērķi, kā arī tiek
uzskaitīti faktori, kas var ierobežot vai ietekmēt programmatūras prasību
specifikāciju.

Otrajā nodaļā tiek norādītas konkrētas prasības, kas satur visu nepieciešamo
programmatūras projektējuma veidošanai. Tā ietver: datu bāzes konceptuālo
modeli, funkcionālās prasības, kas apraksta sistēmas funkciju sadalījumu pa
moduļiem, arējās saskarnes prasības un sistēmas vispārējās prasības.

Trešajā nodaļā tiek aprakstīts projektējums, kas ietver sistēmas sastāvdaļu
aprakstu. Nodaļa satur datu bāzes projektējumu, tās fizisko modeli un daļēju
funkciju un lietotāju saskarņu projektējumu. 

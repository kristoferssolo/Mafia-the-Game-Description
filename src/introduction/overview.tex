\subsection*{Pārskats}

Dokumenta ievads satur ievadinformāciju: dokumenta nolūku, izstrādājamās programmatūras, vispārīgu programmatūras nolūku un funkcijas, saistību ar citiem dokumentiem, kuru prasības tika izmantotas dokumenta izstrādāšanas gaitā un pārskatu par dokumenta daļu saturu ar skaidrojumu dokumenta organizāciju.
Pirmajā nodaļa tiek aprakstīti faktori, kas var ietekmēt produktu un tā prasības.
Nodaļā tiek pamatota programmatūras izstrādes motivācija un nolūks, aprakstītas produkta vieta citu sistēmu perspektīvā, galvenās augsta līmeņa darījumprasības, sistēmas lietotāju grupu lomas un mērķi, kā arī tiek uzskaitīti faktori, kas var ierobežot vai ietekmēt PPS.
Otrajā nodaļā tiek norādītas izstrādājamās programmatūras konkrētas prasības, kas satur visu nepieciešamo programmatūras projektējuma veidošanai.
Tā ietver: datu bāzes konceptuālo modeli, funkcionālās prasības, kas apraksta sistēmas funkciju sadalījumu pa moduļiem, arējās saskarnes prasības un sistēmas vispārējās prasības.
Trešajā nodaļā tiek aprakstīts projektējums, kas ietver sistēmas sastāvdaļu aprakstu turpmākā sistēmas projektējuma atvieglošanai.
Nodaļa satur datu sistēmas bāzes projektējumu un daļēju funkciju un lietotāju saskarņu projektējumu.

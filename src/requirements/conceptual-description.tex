\subsection{Konceptuālais datu bāzes apraksts}
Konceptuālajā modelī redzamās entītijas no konceptuālā ER modeļa (\ref{fig:conceptual_model} attēls):
\begin{itemize}
	\item Lietotājs - reģistrēts lietotājs, kas pieder noteiktai grupai;
	\item Lietotāju grupa - grupa ar saistītp atļauto darbību kopo;
	\item Lietotāju grupas darbība - noteikta darbību sistēmā;
	\item Attēls - datnes metadati un tās adrese, kas ir saistīta ar lietotāju vai spēles lomu;
	\item Maksas abonements - lietotāju maksas abonementa dati. Taču abonementa stāvokļa datu uzglabāšanu atbild maksājuma procesors, no kura ir iegūstama informācija, kā nākamās norēķina datums un vai abonements ir aktīvs.
	\item Spēles uzstādījums - vairāku spēles lomu kopa, kas ir izveidojamas arī publiski (maksas spēlētājiem)
	\item Spēles loma - spēlē izmantojamaās lomas apraksts, katrai lomai obligāti piemī trūkumi un darbības. Tā var tikt izveidota publiski (analoģiski spēles uzstādījumiem);
	\item Lomas darbība - vienai vai vairākām spēles lomas piemītošās spēles darbības apraksts un spēlei specifiskie atribūti(/-s);
	\item Lomas trūkums - vienai vai vairākām spēles lomas piemītošā trūkuma apraksts;
	\item Spēlētājs - vienai virtuālai spēles istabai piederošais spēlētājs. Tam piemīt viena spēles loma un var būt vairākas spēles gaitā veiktās lomai atbilstošās darbības;
	\item Īsziņa - virtuālās istabas terzēšanā izveidotā īsziņa, kas tiek saistīta ar vienu spēlētāju un var atbildēt uz citu īsziņu izveides laikā;
	\item Spēles notikums - spēlē notiekošie notikumi, kā spēles fāzes maiņa, izbalsošanas, slepkavības u.c. .
	\item Spēles vituāla istaba - vienas gaidāmas, tekošās vai pagātnē notikušas spēles, kam piemīt spēlētāji, spēles uzstādījumi, spēles notiukumi, izveidotājs (lietotājs maskas lietotāja grupā);
	\item Paziņojums - universāla (izmantojama iekš un ārpus spēles istabas) paziņojuma dati;
\end{itemize}


\begin{figure}[h]
	\centering
	\includegraphics[width=\linewidth]{./src/requirements/img/conceptual_model.png}
	\caption{Datu bāzes konceptuālais ER modelis}
	\label{fig:conceptual_model}
\end{figure}

\moduleFunctionTable
{Paroles atjaunošanas pieteikums}
{mod-func-auth-pass-restore-app}
{Paroles atjaunošanas pieteikums}
{AMF03}
{
	Funkcijas mērķis ir ļaut lietotājam atjaunot aizmirstu vai nedrošu paroli, nodrošinot drošu paroles maiņas procesu, kas ietver unikāla marķiera izveidi, tā nosūtīšanu lietotāja e-pastā un tā verifikāciju.
}
{
	Ievaddati tiek iegūti no veidlapas.

	Obligātie parametri:
	\begin{enumerate}
		\item E-pasta adrese - simbolu virkne, kas atbilst “RFC 2822: Interneta ziņu formāts” standarta prasībām.
	\end{enumerate}
}
{
	\begin{enumerate}
		\item Pārbauda, vai e-pasta adrese eksistē datubāzē, meklējot lietotāju ar sakrītošu e-pasta adresi;
		      \begin{enumerate}
			      \item Ja tāds lietotājs jau eksistē, parāda 1. paziņojumu. Beidz apstrādi.
		      \end{enumerate}
		\item Ģenerē unikālu marķieri paroles atjaunošanai, pārbaudot unikalitāti, meklējot lietotāju ar sakrītošu un derīgu marķieri;
		      \begin{enumerate}
			      \item Ja tāds lietotājs eksistē, atkārto ģenerāciju līdz iegūtais marķieris ir unikāls.
		      \end{enumerate}
		\item Ieraksta jaunu marķieri lietotāja, kas atjauno paroli, ierakstam, pievienojot tam noteikto derīguma laiku;
		\item Izveido saiti paroles atjaunošanai, iekļaujot marķieri;
		\item Nosūta saiti uz lietotāja e-pasta adresi.
	\end{enumerate}
}
{
	Izvades datu mērķis ir lietotāja informēšana par paroles atjaunošanas pieteikuma ziņas izsūtīšanas stāvokli. Lietotāja saskarnē parādās 4. paziņojums.
	\begin{enumerate}
		Paroles atjaunošanas pieteikuma stāvoklis - kods ar noteiktu stāvokli.
	\end{enumerate}
}
{
	\begin{enumerate}
		\item E-pasta adrese jau ir reģistrēta!;
		\item Saitei ir beidzies derīguma termiņš!;
		\item Parolei ir jāsatur: [neizpildīto paroles prasību saraksts]!;
		\item Apstiprinājuma ziņa ir izsūtīta! Apstipriniet lietotāja kontu ar saiti, kas tiks izsūtīta tuvākā laikā e-pastā.
	\end{enumerate}
}

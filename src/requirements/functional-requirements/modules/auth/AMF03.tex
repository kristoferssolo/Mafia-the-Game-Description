\moduleFunctionTable
{Paroles atjaunošanas pieteikums}
{mod-func-auth-pass-restore}
{Paroles atjaunošanas pieteikums}
{AMF03}
{
	Funkcijas mērķis ir ļaut lietotājam atjaunot aizmirsto vai kompromitēto paroli, nodrošinot drošu paroles maiņas procesu, kas ietver unikāla marķiera izveidi, tā nosūtīšanu lietotāja e-pastā un tā pārbaudi.
}
{
	Obligātie parametri:
	Ievades dati tiek saņemti no ievades lauciņiem, pieejami autentificētiem un neautentificētiem lietotājiem.
	\begin{enumerate}
		\item E-pasta adrese - simbolu virkne, kas atbilst ``RFC 2822: Interneta ziņu formāts'' standarta prasībām.
	\end{enumerate}
}
{
	\begin{enumerate}
		\item Pārbauda, vai e-pasta adrese eksistē datubāzē.
		      Ja nē, parāda 1. paziņojumu.
		\item Ģenerē unikālu marķieri paroles atjaunošanai.
		\item Ieraksta jaunu marķieri lietotāja ierakstam, pievienojot tam derīguma laiku.
		\item Izveido saiti paroles atjaunošanai, iekļaujot marķieri.
		\item Nosūta saiti uz lietotāja e-pasta adresi.
		\item Lietotājs atver saiti.
		\item Pārbauda, vai saitē iekļautais marķieris ir derīgs un nav novecojis.
		      Ja nederīgs vai novecojis, parāda 2. paziņojumu.
		\item Atver paroles atjaunošanas veidlapu.
		\item Lietotājs ievada jauno paroli.
		\item Pārbauda, vai jaunā parole atbilst drošības prasībām.
		      Ja nē, parāda 3. paziņojumu.
		\item Paroles šifrēšanas procesā tiek izmantota jaucējfunkcija ar ``sāls pievienošanu.''
		\item Saglabā jauno paroli datubāzē un parāda 4. paziņojumu.
		\item Atzīmē veco marķieri kā izmantotu.
	\end{enumerate}
}
{
	\begin{enumerate}
		\item Paroles atjaunošanas stāvoklis - kods ar noteiktu stāvokli.
	\end{enumerate}

}
{
	\begin{enumerate}
		\item E-pasta adrese nav reģistrēta!
		\item Saitei ir beidzies derīguma termiņš!
		\item Parolei ir jāsatur: [neizpildīto paroles prasību saraksts]!
		\item Paroles atjaunošana veiksmīga.
	\end{enumerate}
}

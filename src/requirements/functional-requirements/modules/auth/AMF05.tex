\moduleFunctionTable
{Lietotāja pieteikšanās}
{mod-func-auth-login}
{Lietotāja pieteikšanās}
{AMF05}
{
	Autentificēt lietotāju, lai sistēma to uztver kā lietotāju ar konkrēto sistēmas lomu un atļauj turpmākās sistēmas lomas darbības sistēmā.
}
{
	Ievades dati tiek saņemti no pieteikšanās veidlapas.

	Obligātie parametri:
	\begin{enumerate}
		\item E-pasta adrese vai segvārds - simbolu virkne, kas tiek pārbaudīta uz atbilstību e-pasta adreses formātam, kam jāatbilst ``RFC 2822: Interneta ziņu formāts'' standarta prasībām;
		\item Ja tā neatbilst, tad tai jāatbilst sekojošām prasībām: simbolu virkne ar garumu no 6 līdz 20 simboliem, kas var saturēt burtciparu simbolus, skaitļus, defises, pasvītras, apostrofus.
		\item Parole - simbolu virkne ar garumu no 8 līdz 128 simboliem, kas var saturēt burtciparu simbolus, skaitļus, atstarpi, speciālos simbolus: izsaukuma zīmi, dubultpēdiņas, skaitļa zīmi, dolāra zīmi, procenta zīmi, ampersandu, pēdiņas, iekavas, figūriekavas, zvaigznīti, plusu, komatu, mīnusu, punktu, slīpsvītru, kolu, semikolu, salīdzinājuma zīmes, vienādības zīmi, jautājuma zīmi, ``et'' zīmi, slīpsvītru, pasvītru, vertikālo joslu, tildi.
	\end{enumerate}
}
{
	\begin{enumerate}
		\item Pārbauda vai visi lauki ir aizpildīti.
		      \begin{enumerate}
			      \item Ja kāds no laukiem nav aizpildīts, tad parāda 1. paziņojumu.
		      \end{enumerate}
		\item Pārbauda, vai e-pasta adrese vai segvārds un parole satur tikai pieļaujamos simbolus;
		      \begin{enumerate}
			      \item Ja satur, tad iegūst izmantotos neatļautos simbolus, tad parāda 2. paziņojumu ar laukiem un simboliem. Beidz apstrādi.
		      \end{enumerate}
		\item Pārbauda, vai e-pasta adrese vai un parole nepārsniedz noteikto garumu;
		      \begin{enumerate}
			      \item Ja satur, tad iegūst pārsniegto garumu parametru sarakstu un parāda 3. paziņojumu ar laukiem un garumiem. Beidz apstrādi.
		      \end{enumerate}
		\item Iegūst lietotāja autentifikācijas datus no datubāzes, meklējot lietotājus pēc segvārdu vai e-pasta adreses;
		      \begin{enumerate}
			      \item Ja tāds lietotājs netika atrasts, parāda 4. paziņojumu. Beidz apstrādi.
		      \end{enumerate}
		\item Pievieno ievades datu parolei sāls simbolu virkni;
		\item Pārbauda, vai lietotāja sniegtā paroles jaucējfunkcijas rezultāts sakrīt ar datubāzē glabātu vērtību;
		      \begin{enumerate}
			      \item Ja paroles jaucējfunkcijas rezultāts nesakrīt ar datubāzē glabāto vērtību nesakrīt, parāda X paziņojumu. Beidz apstrādi.
		      \end{enumerate}
		\item Ja sakrīt, ģenerē lietotāja sesijas marķieri. Saglabā marķieri kā sīkdatni lietotāja pārlūkprogrammas datu krātuvē.
	\end{enumerate}
}
{
	Izvades datu mērķis ir noteikt, vai lietotājs tiks pāradresēts un kurā lapā lietotājs tiks pāradresēts.
	Lietotāja saskarnē lietotājs tiek pāradresēts uz autentificēto lietotāju sākuma lapu.
	\begin{enumerate}
		\item Paroles atjaunošanas stāvoklis - kods ar noteiktu stāvokli.
	\end{enumerate}
}
{
	\begin{enumerate}
		\item Lauks: [neaizpildīto lauku saraksts] netika aizpildīts (/-i)!;
		\item {}[Parametra nosaukums] nedrīkst saturēt: [izmantoto parametra neatļauto simbolu saraksts]!;
		\item {}[Parametra nosaukums] nedrīkst pārsniegt [noteikto parametra maksimālo simbolu skaits]!;
		\item Lietotājs ar šādu segvārdu vai e-pastu netika atrasts vai parole nav pareiza!
	\end{enumerate}
}

\moduleFunctionTable
{Lietotāja pieteikšanās}
{mod-func-auth-login}
{Lietotāja pieteikšanās}
{AMF05}
{
	Autentificēt lietotāju, lai sistēma to uztver kā lietotāju ar konkrēto sistēmas lomu un atļauj turpmākās sistēmas lomas darbības sistēmā.
}
{
	Ievades dati tiek saņemti no pieteikšanās veidlapas.

	Obligātie parametri:
	\begin{enumerate}
		\item E-pasta adrese vai segvārds - simbolu virkne, kas tiek pārbaudīta uz atbilstību e-pasta adreses formātam, kam jāatbilst ``RFC 2822: Interneta ziņu formāts'' standarta prasībām.
		      Ja tā neabilst, tad tai jāatbilst sekojošām prasībām: simbolu virkne ar garumu no 6 līdz 20 simboliem, kas var saturēt burtciparu simbolus, skaitļus, defises, pasvītras, apostrofus.
		\item Parole - simbolu virkne ar garumu no 8 līdz 128 simboliem, kas var saturēt burtciparu simbolus, skaitļus, atstarpi, speciālos simbolus:
		      izsaukuma zīmi, dubultpēdiņas, skaitļa zīmi, dolāra zīmi, procenta zīmi, ampersandu, pēdiņas, iekavas, figūriekavas,
		      zvaigznīti, plusu, komatu, mīnusu, punktu, slīpsvītru, kolu, semikolu, salīdzinājuma zīmes, vienādības zīmi, jautājuma zīmi, ``at'' zīmi, slīpsvītru, pasvītru, vertikālo joslu, tildi.
	\end{enumerate}
}
{
	\begin{enumerate}
		\item Pārbauda, vai e-pasta adrese vai segvārds un parole satur tikai pieļaujamos simbolus.
		      Ja nesatur, tad iegūst izmantotos neatļautos simbolus, tad parāda 1. paziņojumu ar laukiem un simboliem.
		\item Pārbauda, vai e-pasta adrese vai un parole nepārsniedz noteikto garumu.
		      Ja satur, tad iegūst pārsniegto garumu parametru sarakstu un parāda 2. paziņojumu ar laukiem un garumiem.
		\item Iegūst lietotāja autorizācijas datus no datubāzes, meklējot lietotājus pēc segvārda vai e-pasta adreses.
		      Ja tāds lietotājs netika atrasts, parāda 3. paziņojumu.
		\item Pievieno ievades datu parolei datubāzē glabājamāi sāls simbolu virkni.
		\item Pārbauda, vai lietotāja sniegtā paroles jaucējfunkcijas rezultāts sakrīt ar datubāzē glabātu vērtību.
		      Ja paroles jaucējfunkcijas rezultāts nesakrīt ar datubāzē glabāto vērtību, parāda 4. paziņojumu.
		\item Ja sakrīt, ģenerē lietotāja sesijas marķieri.
	\end{enumerate}
}
{
	Izvades datu mērķis ir noteikt, vai lietotājs tiks pāradresēts un kurā lapā lietotājs tiks pāradresēts, un ierakstīt sesijas marķieri lietotāja pārlūkprogrammas datu krātuvē.
	\begin{enumerate}
		\item Reģistrācijas apstiprinājuma stāvoklis - kods ar noteiktu stāvokli.
		\item Lietotāja sesijas marķieris.
	\end{enumerate}
}
{
	\begin{enumerate}
		\item {}[Parametra nosaukums] nedrīkst saturēt: [izmantoto parametra neatļauto simbolu saraksts]!
		\item {}[Parametra nosaukums] nedrīkst pārsniegt [noteikto parametra maksimālo simbolu skaits]!
		\item {}Lietotājs ar šādu segvārdu vai e-pastu netika atrasts vai parole nav pareiza!
	\end{enumerate}
}

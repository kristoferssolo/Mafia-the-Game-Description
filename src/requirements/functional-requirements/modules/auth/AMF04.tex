\moduleFunctionTable
{Paroles atjaunošana}
{mod-func-auth-pass-restore}
{Paroles atjaunošana}
{AMF04}
{
	Funkcijas mērķis ir ļaut lietotājam atjaunot aizmirstu vai nedrošu paroli, nodrošinot drošu paroles maiņas procesu, kas ietver unikāla marķiera izveidi, tā nosūtīšanu lietotāja e-pastā un tā pārbaudi.
}
{
	Ievaddati tiek iegūti no apstiprinājuma vietrādes parametriem ar kuru lietotājs piekļūst sistēmas funkcijai.

	Obligātie parametri:
	\begin{enumerate}
		\item Lietotāja paroles atjaunošanas marķieris - 16 simbolu gara virkne, kas tiek iegūta no saites parametriem;
		\item Lietotāja jaunā parole - atbilst \hyperref[tab:IIDP06]{IIDP06};
		\item Lietotāja jaunās parole apstiprinājums - atbilst \hyperref[tab:IIDP06]{IIDP06}.
	\end{enumerate}
}
{
	\item Ja lietotāja paroles identifikatora atjaunošanas marķieris nav iesniegts, parāda 1. paziņojumu. Beidz apstrādi;
	\item Sameklē lietotāja ierakstu datubāzē, meklējot to pēc atjaunošanas marķiera;
	\begin{enumerate}
		\item Ja lietotāja ieraksts netika atrasts, parāda 2. paziņojumu. Beidz apstrādi;
		\item Pārbauda, vai saitē iekļautais marķieris ir derīgs un nav novecojis. Ja tas ir nederīgs vai novecojis, parāda 2. paziņojumu. Beidz apstrādi.
	\end{enumerate}
	\item Pārbauda, vai parole un paroles apstiprinājums ir iesniegts;
	\begin{enumerate}
		\item Ja kāds no laukiem nav iesniegts parāda 4. paziņojumu ar attiecīgo lauku nosaukumiem.
	\end{enumerate}
	\item Pārbauda, vai jaunā parole atbilst drošības prasībām;
	\begin{enumerate}
		\item Ja nē, parāda 5. paziņojumu ar neizpildīto prasību sarastu. Beidz apstrādi.
	\end{enumerate}
	\item Pievieno parolei sāls simbolu virkni, šifrē paroli ar jaucējfunkciju;
	\item Ieraksta marķiera derīguma termiņu pamaina uz tagadējo laiku;
	\item Atjaunoto lietotāja ierakstu ieraksta datubāzē.
	\begin{enumerate}
		\item Ja ierakstīšana neizdevās, parāda 3. paziņojumu.
	\end{enumerate}
}
{
	Izvades datu mērķis ir lietotāja informēšana par paroles atjaunošanas stāvokli. Lietotāja saskarnē parādās 6. paziņojums.
	\begin{enumerate}
		\item Paroles atjaunošanas stāvoklis - kods ar noteiktu stāvokli.
	\end{enumerate}
}
{
	\begin{enumerate}
		\item Paroles atjaunošanas marķieris nav norādīts!
		\item Paroles atjaunošanas marķieris nav derīgs!
		\item Sistēmas kļūda! Mēģiniet vēlreiz vēlāk!
		\item Lauks: [neaizpildīto lauku saraksts] netika aizpildīts (/-i)!
		\item Parolei ir jāsatur: [neizpildīto paroles prasību saraksts]!
		\item Paroles atjaunošana ir veiksmīga!
	\end{enumerate}
}

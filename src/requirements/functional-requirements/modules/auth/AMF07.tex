\moduleFunctionTable
{Lietotāja konta apstiprināšana}
{mod-func-auth-email-confirm}
{Lietotāja konta apstiprināšana}
{AMF07}
{
	Funkcijas mērķis ir apstiprināt lietotāja konta e-pasta adrese, i.e., apstiprināt to, ka lietotājam pieder tā norādītā e-pasta adrese.
}
{
	Ievaddati tiek iegūti no apstiprinājuma vietrādes parametriem, ar kuras lietotājs piekļūst sistēmas funkcijai.

	Obligātie parametri:
	\begin{enumerate}
		\item E-pasta apstiprinājuma marķieris - 16 simbolu gara virkne, kas tiek iegūta no saites parametriem.
	\end{enumerate}

}
{
	\begin{enumerate}
		\item Pārbauda, vai ievades datos ir e-pasta apstiprinājuma marķieris.
		      Ja tā nav, tad parāda 1. paziņojumu.
		\item Pārbauda, vai e-pasta adreses apstiprinājuma marķieris atbilst sagaidāmam garumam.
		      Ja neatbilst, parāda 2. paziņojumu.
		\item Meklē datubāzē lietotājus ar iesniegto marķieri.
		      Ja tāds lietotājs netiek atrasts, parāda 3. paziņojumu.
		      Ja datubāzē atrastā lietotāja e-pasta apstiprināšanas karodziņš apzīmē apstiprinātu e-pastu, parāda 4. paziņojumu.
		\item Attiecīgā datubāzes ierakstam pamaina karodziņa stāvokli uz apstiprinātu.
	\end{enumerate}
}
{
	Izvades datu mērķis ir noteikt, vai lietotājs tiks pāradresēts.

	\begin{enumerate}
		\item E-pasta adreses apstiprinājuma stāvoklis - kods ar noteiktu stāvokli.
	\end{enumerate}

}
{
	\begin{enumerate}
		\item Apstiprināšanas saite nav korekta: marķieris nav norādīts! Mēģiniet vēlreiz vai pieprasiet atkārtotu apstiprinājuma ziņas izsūtīšanu!
		\item Apstiprināšanas saite nav korekta: marķieris nav korekts! Mēģiniet vēlreiz vai pieprasiet atkārtotu apstiprinājuma ziņas izsūtīšanu!
		\item Marķieris nav aktuāls vai nav korekts! Mēģiniet vēlreiz vai pieprasiet atkārtotu apstiprinājuma ziņas izsūtīšanu!
		\item E-pasts jau ir apstiprināts!
	\end{enumerate}
}

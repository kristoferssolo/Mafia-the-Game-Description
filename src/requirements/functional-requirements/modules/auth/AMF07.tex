\moduleFunctionTable
{Lietotāja konta apstiprināšana}
{mod-func-auth-email-confirm}
{Lietotāja konta apstiprināšana}
{AMF07}
{
	Funkcijas mērķis ir apstiprināt lietotāja konta e-pasta adresi, i.e., apstiprināt to, ka lietotājam pieder norādītā e-pasta adrese.
}
{
	Ievaddati tiek iegūti no apstiprinājuma vietrādes parametriem, ar kuras lietotājs piekļūst funkcijai.

	Obligātie parametri:
	\begin{enumerate}
		\item E-pasta apstiprinājuma marķieris - 16 simbolu gara virkne, kas tiek iegūta no saites parametriem.
	\end{enumerate}

}
{
	\item Pārbauda, vai ievades datos ir e-pasta apstiprinājuma marķieris;
	\begin{enumerate}
		\item Ja tā nav, tad parāda 1. paziņojumu. Beidz apstrādi.
	\end{enumerate}
	\item Pārbauda, vai e-pasta adreses apstiprinājuma marķieris atbilst sagaidāmam garumam;
	\begin{enumerate}
		\item Ja neatbilst, parāda 2. paziņojumu. Beidz apstrādi.
	\end{enumerate}
	\item Meklē datubāzē lietotājus ar iesniegto marķieri;
	\begin{enumerate}
		\item Ja tāds lietotājs netiek atrasts, parāda 2. paziņojumu. Beidz apstrādi;
		\item Ja datubāzē atrastā lietotāja e-pasta apstiprināšanas karodziņš apzīmē apstiprinātu e-pastu, parāda 3. paziņojumu. Beidz apstrādi.
	\end{enumerate}
	\item Ierakstam pamaina karodziņa stāvokli uz kodu, kas atbilst apstiprinātam stāvoklim un rediģētu ierakstu ieraksta datubāzē.
}
{
	Izvades datu mērķis ir lietotāja informēšana par konta apstiprināšanas stāvokli. Lietotājam tiek parādīts 4. paziņojums.
	\begin{enumerate}
		\item E-pasta adreses apstiprinājuma stāvoklis - kods ar noteiktu stāvokli.
	\end{enumerate}
}
{
	\begin{enumerate}
		\item Apstiprināšanas saite nav korekta: marķieris nav norādīts! Mēģiniet vēlreiz vai pieprasiet atkārtotu apstiprinājuma ziņas izsūtīšanu!;
		\item Marķieris nav aktuāls vai nav korekts! Mēģiniet vēlreiz vai pieprasiet atkārtotu apstiprinājuma ziņas izsūtīšanu!;
		\item E-pasts jau ir apstiprināts!;
		\item E-pasts ir veiksmīgi apstiprināts!
	\end{enumerate}
}

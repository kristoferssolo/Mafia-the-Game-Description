\moduleFunctionTable
{Apstiprinājuma ziņas atkārtotās izsūtīšanas pieteikums}
{mod-func-auth-app}
{Apstiprinājuma ziņas atkārtotās izsūtīšanas pieteikums}
{AMF02}
{
	Funckijas mērķis ir izsūtīt e-pasta apstiprinājumu atkārtoti lietotājam, kas jau veica reģistrāciju
}
{
	Ievades dati tiek iegūti no veidlapas datiem, kas tiek aizpildīta automātiski, ja lietotājs veic reģistrācijas procesu.

	Obligātie parametri:
	\begin{enumerate}
		\item Lietotāja identifikators - sistēmā izmantots lietotāja identifikators - vesels pozitīvs skaitlis skaitlis, kas atbilst datubāzē glabājamā skaitliska identifikatora diapazonam.
	\end{enumerate}
}
{
	\begin{enumerate}
		\item Pārbauda, vai identifikators atbilst paredzētajam datu tipam.
		      Ja neatbilst parāda 1. paziņojumu.
		\item Meklē lietotāju datubāzē pēc ievades datu identifikatora parametra.
		      Ja tāds lietotājs neeksistē, parāda 1. paziņojumu.
		\item Iegūst no datubāzes atrastā lietotāja tā e-pasta adresi.
		\item Iegūst atrastā lietotāja tā e-pasta adresi.
		\item Ģenerē e-pasta apstiprinājuma marķieri.
		      Izveido saiti apstiprinājumam, iekļaujot e-pasta apstiprinājuma marķieri.
		\item Sagatavo e-pasta ziņas saturu no šablona, ievietojot tajā apstiprinājuma saiti.
		\item Pieprasa e-pasta aizsūtīšanu.
		      Ja tā netiek apstiprināta, parāda 1. paziņojumu.
		\item Ja e-pasta ziņas aizsūtīšana ir apstiprināta, parāda 2. paziņojumu.
	\end{enumerate}
}
{
	Izvades datu mērķis ir noteikt, vai lietotājs tiks pāradresēts un kurā lapā lietotājs tiks pāradresēts.
	\begin{enumerate}
		\item E-pasta apstiprinājuma ziņas izsūtīšanas stāvoklis - kods ar noteiktu stāvokli.
	\end{enumerate}
}
{
	\begin{enumerate}
		\item Sistēmas iekšējā kļūda! Mēģiniet pārlādēt lapu vai mēģiniet vēlāk!
		\item Apstiprinājuma ziņa tiks izsūtīta 1-10 minūšu laikā.
	\end{enumerate}
}

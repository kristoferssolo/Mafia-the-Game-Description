\moduleFunctionTable
{Lietotāju profilu pārskats}
{mod-func-user-profiles}
{Lietotāju profilu pārskats}
{LAMF01}
{
	Funkcijas mērķis ir lietotājiem sniegt citu lietotāju profilu publisku informāciju. Funkcijas ietvaros var tikt veikta neobligāta meklēšana pēc noteiktiem lietotāja profilu atribūtiem.
}
{
	Ievades datus iegūst no lietotāja neobligāti uzstādītiem filtriem, kārtošanas izvēles un lappuses numura un meklēšanas uzvednes.
	Parametri atbilst attiecīgām saitēm un ievades laukam lietotāja saskarnē un izvēlnēm pārskata lapās.
	Parametru vērtības tiek iegūtas no vietrādes parametriem.

	Neobligātie parametri:
	\begin{enumerate}
		\item Lappuses numurs - vesels pozitīvs skaitlis;
		\item Meklēšanas uzvedne - simbolu virkne ar garumu līdz 50 simboliem bez atļauto simbolu ierobežojumiem;
		\item Kārtošanas vārdnīcu saraksts, kas sastāv no vārdnīcām: datu bāzes atribūta nosaukums (atbilst \hyperref[tab:IIDP12]{IIDP12}) - kārtošanas kods (atbilst \hyperref[tab:IIDP11]{IIDP11});
		\item Filtru vārdnīcu saraksts, kas sastāv no vārdnīcām: datu bāzes atribūta nosaukums (atbilst \hyperref[tab:IIDP12]{IIDP12}) - filtra vērtība (vesels skaitlis) un filtra veids (0 - Būla mainīgā filtrs, 1 - entitātes identifikatora filtrs).
	\end{enumerate}
}
{
	\begin{enumerate}
		\item Sāk gatavot datubāzes pieprasījumu no lietotāju tabulas;
		\item Sagatavo pārskata pieprasījumu un iegūst lappuses numuru un kopējo lappušu skaitu, izmantojot \hyperref[tab:KPR07]{KPR07} ar lappuses numuru, meklēšanas uzvedni, kārtošanas vārdnīcu sarakstu, filtru vārdnīcu sarakstu, šos parametrus iesniedzot, ja tie ir iesniegti funkcijā;
		      \begin{enumerate}
			      \item Ja lappušu skaits ir 0, tad parāda 1. paziņojumu. Beidz apstrādi.
		      \end{enumerate}
		\item Veic sagatavoto pieprasījumu, iegūstot lietotāja segvārdu, lietotāja konta izveidošanas laiku (lietotāja pievienošanās laiku), lietotāja attēla datnes adreses;
		      \begin{enumerate}
			      \item Ja pieprasījums neizdodas, parāda 2. paziņojumu. Beidz apstrādi.
		      \end{enumerate}
		\item Katram ierakstam no rezultāta, iegūst sameklē attēla datnes adresi no attēlu tabulas pēc attēla identifikatora.
		      \begin{enumerate}
			      \item Ja attēla datnes adrese neeksistē ierakstam, tad iegūst noklusētā attēla datnes adresi un pamaina ieraksta datnes adresi izvades datos uz noklusētā datnes attēla adresi.
		      \end{enumerate}
	\end{enumerate}
}
{
	Izvades datu mērķis ir parādīt rezultāta pārskatu lietotāja saskarnē.
	\begin{enumerate}
		\item Pārskata ierakstu saraksts: ierakstu saraksts, kas sastāv no: lietotāja identifikatora, segvārda, lietotāja konta izveidošanas laika (lietotāja pievienošanās laiku);
		\item Kopējais lapu skaits - vesels pozitīvs skaitlis;
		\item Tekošā lappuse - vesels pozitīvs skaitlis, kas ir mazāks vai vienāds par lapu skaitu;
		\item Kārtošanas vārdnīcu saraksts, kas sastāv no vārdnīcām: datu bāzes atribūta nosaukums (atbilst \hyperref[tab:IIDP12]{IIDP12}) - kārtošanas kods (atbilst \hyperref[tab:IIDP11]{IIDP11});
		\item Filtru vārdnīcu saraksts, kas sastāv no vārdnīcām: datu bāzes atribūta nosaukums (atbilst \hyperref[tab:IIDP12]{IIDP12}) - filtra vērtība (vesels skaitlis) un filtra veids ($0$ - Būla mainīgā filtrs, $1$ - entitātes identifikatora filtrs).
	\end{enumerate}
}
{
	\begin{enumerate}
		\item Netika atrasts neviens lietotājs!;
		\item Sistēmas iekšējā kļūda! Mēģiniet vēlreiz vēlāk!
	\end{enumerate}
}

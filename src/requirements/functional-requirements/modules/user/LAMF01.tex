\moduleFunctionTable
{Lietotāju profilu pārskats}
{mod-func-user-profiles}
{Lietotāju profilu pārskats}
{LAMF01}
{
	Funkcijas mērķis ir lietotājiem rādīt citu lietotāju profilu publisku informāciju (profilu).
	Funkcijas ietvaros var tikt veikta neobligāta meklēšana pēc noteiktiem lietotāja profilu atribūtiem.
}
{
	Ievades datus iegūst no veicamās darbības.

	Neobligātie parametri:
	\begin{enumerate}
		\item Meklēšanas uzvedne - simbolu virkne ar garumu līdz 50 simboliem bez atļauto simbolu ierobežojumiem.
	\end{enumerate}
}
{
	\begin{enumerate}
		\item Sagatavo datubāzes pieprasījumu no lietotāju tabulas.
		\item Ja meklēšanas uzvedne nav tukša simbolu virkne, tad pieprasījumam pievieno meklēšanas nosacījumu meklēšanai pēc pilna vārda, segvārda un biogrāfijas.
		\item Pieprasa ierakstu saskaitīšanu, izmantojot sagatavoto pieprasījumu.
		      Ja rezultātu skaits ir $0$, tad parāda 1. paziņojumu.
		\item Aprēķina lappušu skaitu ar formulu: $L = ceil(Q / Q_l)$, $Q$ - rezultātu skaits, $Q_l$ - ierakstu skaits vienā lappusē.
		\item Ja ievaddatos nav lappuses numurs vai tā pārsniedz aprēķināto lappušu skaitu, tad turpmāk lappuses numurs būs 1.
		\item Aprēķina ierakstu nobīdi ar formulu: $O = (N - 1) * Q_l$, kur $O$ - nobīde; $N$ - lappuses numurs, $Q_l$ - ierakstu skaits vienā lappusē.
		\item Pievieno pieprasījumam nobīdi pēc aprēķinātās lappuses.
		\item Veic sagatavoto pieprasījumu, iegūstot abonementa segvārdu, izveidošanas laiku (lietotāja pievienošanās laiku), attēla datnes adreses.
		      Ja pieprasījums neizdodas, parāda 2. paziņojumu.
		\item Ja attēls datnes adrese neeksistē, tad iegūst noklusētā attēla datnes adresi.
	\end{enumerate}
}
{
	Izvades datu mērķis ir parādīt pārskata, ņemot meklēšanas uzvedni, ja tā tukša.

	Izvades dati:
	\begin{enumerate}
		\item Pārskata ieraksta dati: vairāki ieraksti, kas sastāv no segvārda, lietotāja konta izveidošanas laika (lietotāja pievienošanās laiku).
		\item Lapu skaits - pozitīvs skaitlis.
		\item Tekošā lappuse - pozitīvs skaitlis, kas ir mazāks vai vienāds par lapu skaitu.
	\end{enumerate}
}
{
	\begin{enumerate}
		\item Pēc jūsu meklēšanas uzvednes netika atrasts neviens lietotājs!
		\item Notika sistēmas iekšējā kļūda! Mēģiniet vēlreiz vēlāk!
	\end{enumerate}
}

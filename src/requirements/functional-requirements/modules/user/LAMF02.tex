\moduleFunctionTable
{Lietotāja konta detaļas}
{mod-func-user-profile-data}
{Lietotāja konta detaļas}
{LAMF02}
{
	Funkcijas mērķis ir reģistrētiem lietotājiem saņemt savu informāciju, kas ietver gan publisko informāciju, gan privāto. Administratoriem informācija ir iegūstama par jebkuru lietotāju.
}
{
	Ievades datus iegūst no vietrāža parametriem, caur kuru tiek piekļūts funkcijai. Alternatīvi, dati (lietotāja identifikators) tiek iegūti no konteksta.

	Neobligātie parametri:
	\begin{enumerate}
		\item Apskatāmo konta datu lietotāja identifikators - atbilst IIDP10. Noklusētā vērtība - no konteksta lietotāja (kas piekļūst funkciju) iegūtais identifikators.
	\end{enumerate}
}
{
	\begin{enumerate}
		\item Sagatavo datubāzes pieprasījumu no lietotāju tabulas;
		\item Iegūst lietotāja lomu sistēmā;
		      \begin{enumerate}
			      \item Ja lietotājs nav administrators, pieprasījumam pievieno nosacījumu, ka abonementa ierakstiem jābūt saistītiem ar lietotāju.
		      \end{enumerate}
		\item Ja apskatāmo datu kontu lietotāja identifikators sakrīt ar lietotāja identifikatoru, tad parāda X paziņojumu. Beidz apstrādi;
		\item Pieprasījumam pievieno atlasīšanu pēc lietotāja identifikatora;
		\item Sagatavo pieprasīto lauku sarakstu. Saraksta pamatā ir pilns vārds, segvārds, biogrāfijas informācija, konta izveidošanas laiks, attēls (datnes adrese);
		      \begin{enumerate}
			      \item Ja pieprasītājs lietotājs ir administrators, tad pie šī saraksta pievieno arī e-pasta adresi, e-pasta apstiprinājuma informāciju, konta stāvokli.
		      \end{enumerate}
		\item Veic sagatavoto pieprasījumu, pieprasot iepriekš sagatavoto lauku sarakstu, attēla datnes adreses;
		      \begin{enumerate}
			      \item Ja pieprasījums neizdodas, parāda 2. paziņojumu. Beidz apstrādi.
			      \item Ja lietotājs netika atrasts, parāda 3. paziņojumu. Beidz apstrādi.
		      \end{enumerate}
		\item Ja attēla datnes adrese neeksistē, tad iegūst noklusētā attēla datnes adresi.
	\end{enumerate}
}
{
	Izvades datu mērķis ir lietotāja konta datu parādīšana.
	Ja pieprasītājs lietotājs nav administrators:
	\begin{enumerate}
		\item Vārdnīca:
		      \begin{enumerate}
			      \item pilns vārds - simbolu virkne;
			      \item segvārds - simbolu virkne;
			      \item biogrāfijas informācija - simbolu virkne;
			      \item konta izveidošanas laiks - datums simbolu virknes formātā;
			      \item attēls - datnes adrese
		      \end{enumerate}
		\item Papildus vārdnīcas dati, ja pieprasītājs lietotājs ir administrators:
		      \begin{enumerate}
			      \item e-pasta adrese - IIDP05;
			      \item e-pasta apstiprinājuma stāvoklis - skaitlisks kods (0 - neapstiprināts; 1- apstiprinājuma vēstule ir aizsūtīta; 1 - apstiprināts );
			      \item konta stāvoklis - IIDP10
		      \end{enumerate}
	\end{enumerate}
}
{
	\begin{enumerate}
		\item Darbība nav autorizēta!
		\item Sistēmas iekšējā kļūda! Mēģiniet vēlreiz vēlāk!
		\item Tāds lietotājs nav atrasts! Mēģiniet vēlreiz!
	\end{enumerate}
}

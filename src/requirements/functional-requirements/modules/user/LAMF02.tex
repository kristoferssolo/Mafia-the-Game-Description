\moduleFunctionTable
{Lietotāja konta detaļas}
{mod-func-user-profile-data}
{Lietotāja konta detaļas}
{LAMF02}
{
	Funkcijas mērķis ir reģistrētiem lietotājiem iegūt konta savu informāciju, kas ietver gan publisko informāciju, gan privāto.
	Administratoriem informācija ir iegūstamā par jebkuru lietotāju.
}
{
	Ievades datus iegūst no vietrāža parametriem, caur kuru tiek piekļūts funkcijai.
	Alternatīvi, dati tiek iegūti no konteksta.

	Obligātie parametri:
	\begin{enumerate}
		\item Konta datu lietotāja identifikators - pozitīvs skaitlis.
		      Noklusētā vērtība - no konteksta lietotāja (kas piekļūst funkciju) iegūtais identifikators.
	\end{enumerate}
}
{
	\begin{enumerate}
		\item Sagatavo datubāzes pieprasījumu no lietotāju tabulas.
		\item Iegūst lietotāja lomu sistēmā.
		      Ja lietotājs nav administrators, pieprasījumam pievieno nosacījumu, ka abonementa ierakstiem jābūt saistītiem ar lietotāju.
		\item Ja apskatāmo datu kontu lietotāja identifikators nesakrīt ar lietotāja identifikatoru, tad parāda 1. paziņojumu.
		\item Pieprasījumam pievieno atlasīšanu pēc lietotāja identifikatora.
		\item Sagatavo pieprasīto lauku sarakstu.
		      Saraksta pamatā ir pilns vārds, segvārds, biogrāfijas informācija, konta izveidošanas laiks, attēls (datnes adrese).
		      Ja pieprasītajā lietotājs ir administrators, tad pie šī saraksta pievieno arī e-pasta adresi, e-pasta apstiprinājuma informāciju, konta stāvokli.
		\item Veic sagatavoto pieprasījumu, pieprasot iepriekš sagatavoto lauku sarakstu, attēla datnes adreses.
		      Ja pieprasījums neizdodas, parāda 2. paziņojumu.
		      Ja lietotājs netika atrasts, parāda 3. paziņojumu.
		\item Ja attēls datnes adrese neeksistē, tad iegūst noklusētā attēla datnes adresi.
	\end{enumerate}
}
{
	Izvades datu mērķis ir lietotāja konta datu izvadīšana.
	To saturs ir atkarīgs no pieprasītāja lietotāja lomas sistēmā.

	\begin{enumerate}
		\item Vārdnīca - pilns vārds - simbolu virkne, segvārds - simbolu virkne, biogrāfijas informācija - simbolu virkne, konta izveidošanas laiks - datums simbolu virknes formātā, attēls - datnes adrese.
		      Ja pieprasītais lietotājs ir administrators, tad vārdnīca ir arī e-pasta adrese - simbolu virkne, e-pasta apstiprinājuma stāvoklis - simbolu virkne, konta stāvoklis - simbolu virkne.
	\end{enumerate}

}
{
	\begin{enumerate}
		\item Darbība nav autorizēta!
		\item Notika sistēmas iekšējā kļūda! Mēģiniet vēlreiz vēlāk!
		\item Tāds lietotājs nav atrasts! Mēģiniet vēlreiz!
	\end{enumerate}
}

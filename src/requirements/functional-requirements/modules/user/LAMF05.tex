\moduleFunctionTable
{Lietotāja konta dzēšana}
{mod-func-user-delete}
{Lietotāja konta dzēšana}
{LAMF05}
{
	Funkcijas mērķis ir dzēst lietotāju kontus, lai to konta informācija būtu neatgriezeniski izdzēsta.
}
{
	Ievades dati tiek saņemti no reģistrēto lietotāju pieejamās darbības.
	Alternatīvi, dati tiek iegūti no konteksta (autentificēta lietotāja identifikators).

	Obligātie parametri:
	\begin{enumerate}
		\item Konta datu lietotāja identifikators - pozitīvs skaitlis.
		      Noklusētā vērtība - no konteksta lietotāja (kas piekļūst funkciju) iegūtais identifikators.
	\end{enumerate}
}
{
	\begin{enumerate}
		\item Ja lietotājs nav administrators, parādīt 1. paziņojumu.
		\item Pārbauda, vai lietotājs ar tādu identifikatoru eksistē.
		      Ja neeksistē, parāda 2. paziņojumu.
		\item Lietotāja ierakstu izdzēš.
		      Ja izdzēšana ir veiksmīga, parāda 3. paziņojumu.
		      Ja izdzēšana nav veiksmīga, parāda 4. paziņojumu.
	\end{enumerate}
}
{
	Izvades datu mērķis ir noteikt, vai lietotājs tiks pāradresēts.
	\begin{enumerate}
		\item Lietotāja konta dzēšanas stāvoklis - kods ar noteiktu stāvokli.
	\end{enumerate}
}
{
	\begin{enumerate}
		\item Darbība nav autorizēta!
		\item Tāds lietotājs nav atrasts! Mēģiniet vēlreiz!
		\item Sistēmas iekšējā kļūda! Mēģiniet vēlreiz!
		\item Lietotāja deaktivizēšana ir veiksmīga!
	\end{enumerate}
}

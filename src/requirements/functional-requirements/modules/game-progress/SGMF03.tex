\moduleFunctionTable
{Spēles notikumu pārskats}
{mod-func-progress-overview}
{Spēles notikumu pārskats}
{SGMF03}
{
	Funkcijas mērķis ir iegūt informāciju par spēles notikumiem.
}
{
	Obligātie parametri:
	\begin{enumerate}
		\item Spēles istabas identifikators - vesels pozitīvs skaitlis.
		\item Paslēptā informācija - karodziņš.
	\end{enumerate}
}
{
	\begin{enumerate}
		\item Meklē spēles istabu attiecīgā tabulā, izmantojot darbības identifikatoru.
		      Ja spēles istaba netiek atrasta parāda 1. paziņojumu.
		\item Ja pieprasītājs nav sistēma un ir pieprasīta paslēptā informācija un spēles istabas stāvoklis neapzīmē pabeigtu spēli, parāda 2. paziņojumu.
		\item Sāk veidot sarakstu ar notikumu vārdnīcām.
		\item No datubāzes iegūst spēles notikumu ierakstus.
		      Katram notikumam iegūst veidu.
		\item Ja notikuma redzamības spēles procesā ir patiess, tad pievieno attiecīgo vārdnīcu sarakstam.
	\end{enumerate}
}
{
	Izvades mērķis ir par notikumiem nepieciešamās informācijas apkopošana vārdnīcā atkarībā no paslēptās informācijas ievades karodziņa.
	Ja tiek izvadīta paslēptā informācija, saraksts sastāvēs no vārdnīcām, kas sastāv no:
	\begin{enumerate}
		\item Notikuma veids - skaitlisks kods;
		\item Ietekmēto spēlētāju saraksts - veselu pozitīvu skaitļu saraksts.
	\end{enumerate}
	Ja paslēptā informācija izvadīta netiks, tad spēlētāju vārdnīca sastāvēs tikai no lomas identifikatora.
}
{
	\begin{enumerate}
		\item Spēles istaba ar identifikatoru [istabas identifikators] netika atrasta!
		\item Notikumu detalizēts pārskats nav pieejams spēles laikā!
	\end{enumerate}
}

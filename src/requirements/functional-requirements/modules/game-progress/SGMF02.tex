\moduleFunctionTable
{Spēles notikuma izveidošana}
{mod-func-progress-create}
{Spēles notikuma izveidošana}
{SGMF02}
{
	Funkcijas mērķis ir izveidot visiem spēlētājiem aktuālu spēles notikumu.
	Notikumi galvenokārt ir spēlētāju izraisīti.
	Taču daļa no notikumiem ir atkarīgi no laika un fāzes.
}
{
	Ievades dati tiek iegūti no esošās spēles procesa konteksta.
	\begin{enumerate}
		\item izveidošanas laiks - simbolu virkni ar datumu noteiktā formatējumā;
		\item vai ir nakts - karodziņš;
		\item darbības identifikators - vesels pozitīvs skaitlis vai 0;
		\item spēlētāja identifikators - vesels pozitīvs skaitlis vai 0.
	\end{enumerate}
}
{
	\begin{enumerate}
		\item Meklē spēles istabu attiecīgā tabulā, izmantojot darbības identifikatoru.
		      Ja spēles istaba netiek atrasta parāda 1. paziņojumu.
		\item Meklē spēles darbību datubāzē, izmantojot darbības identifikatoru.
		      Ja spēles istaba netiek atrasta parāda 1. paziņojumu.
		\item Pārbauda, vai datuma formāts ir korekts.
		      Ja nav korekts, parāda X paziņojumu.
		\item Iegūst vēlāko notikumu nakts maiņas datubāzē.
		      Ja netika atrasts neviens, tad pieņem, ka ir 0-tā nakts (neviena nakts vēl nav notikusi).
		\item Pārbauda, vai spēles notikumu identifikatori datubāzes ierakstā sakrīt ar sarakstā atrodamiem.
		\item Izveido jaunu notikuma ierakstu datubāzē, izmantojot sagatavotos datus.
		      Ja izveidošana nenotika parāda 3. paziņojumu.
	\end{enumerate}
}
{
	Izvades mērķis ir apstiprināt notikuma izveidošanu.
	\begin{enumerate}
		\item Notikumu izveidošanas stāvokļa kods - skaitlisks kods.
	\end{enumerate}
}
{
	\begin{enumerate}
		\item Spēles istaba ar identifikatoru [istabas identifikators] netika atrasta!
		\item Darbība ar identifikatoru [istabas identifikators] netika atrasta!
		\item Notikuma izveidošana neizdevās: nevar ierakstīt datubāzē!
	\end{enumerate}
}

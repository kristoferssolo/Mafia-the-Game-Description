\moduleFunctionTable
{Spēles darbības veikšana}
{mod-func-progress-action}
{Spēles darbības veikšana}
{SGMF01}
{
	Funkcijas mērķis ir veikt spēles darbību lietotājiem, kas atrodas spēles istabā, saglabājot darbībau kā notikuma ierakstu.
}
{
	Ievade tiek iegūta no darbības spēles laikā.

	Obligātie parametri:
	\begin{enumerate}
		\item Darbības identifikators - vesels pozitīvs skaitlis.
		\item Spelētāja identifikators - vesels pozitīvs skaitlis.
		\item Veicēja lietotāja identifikators - vesels pozitīvs skaitlis.
		\item Spēles istabas identifikators - vesels pozitīvs skaitlis.
	\end{enumerate}

	Neobligātie parametri:
	\begin{enumerate}
		\item Mērķa spēlētāju identifikatori - veselu pozitīvu skaitļu saraksts.
		      Noklusētā vērtība: tukšs saraksts.
	\end{enumerate}
}
{
	\begin{enumerate}
		\item Meklē spēles istabu attiecīgā tabulā, izmantojot darbības identifikatoru.
		      Ja spēles istaba netiek atrasta parāda 1. paziņojumu.
		      Ja speles istabas stāvoklis neatbilst stāvoklim spēlei procesā, tad parāda 2. paziņojumu.
		\item Meklē spēles darbību attiecīgā tabulā, izmantojot darbības identifikatoru.
		      Ja darbība netiek atrasta parāda 2. paziņojumu.
		\item Ja mērķa spēlētāju saraksts nav tukšs, meklē spēlētāju attiecīgajā tabulā, izmantot spēlētāja identifikatorus.
		      Ja viens no mērķa spēlētājiem netiek atrasts parāda 2. paziņojumu.
		\item Meklē mērķa spēlētājus attiecīgajā tabulā, izmantot spēlētāja identifikatoru.
		      Ja darbība netiek atrasta parāda 3. paziņojumu.
		\item Iegūst lietotāja identifikatoru no atrastā ieraksta.
		      Pārbauda vai lietotāja identifikators sakrīt ar spēlētāja identifikatoru.
		      Ja nesakrīt, parāda 2. paziņojumu.
		\item Iegūst lomas identifikatoru no atrastā ieraksta.
		      Meklē lomu, izmantojot identifikatoru.
		      Ja loma netiek atrasta, parāda 2. paziņojumu.
		\item Pārbauda, vai darbība ir ar lomu saistīta darbība.
		      Ja tā nav, parāda 3. paziņojumu.
		\item Meklēt spēles notikumus datubāzē, noskaidro vai pēdējais pēc izveidošanas notikums aizliedz darbību vai to atļauj.
		      Ja pēdējais nav atļaujošais notikums, parāda 2. paziņojumu.
		\item Izveido jaunu ierakstu darbību.
		      Ja ieraksta veikšana neizdevās.
		      Parāda 2. paziņojumu.
	\end{enumerate}
}
{
	Izvades mērķis ir parādīt ar izpildīto darbību saistīto informāciju.
	\begin{enumerate}
		\item Darbības veikšanas apstiprinājuma teksts - simbolu virkne.
		\item Darbības stāvokļa kods - skaitlisks kods.
	\end{enumerate}
}
{
	\begin{enumerate}
		\item Spēle ir beigusies, nevar veikt darbību!
		\item Sistēmas kļūda! Mēģiniet vēlreiz.
		\item Darbība nav atļauta jūsu lomai!
		\item Darbība nav atļauta šajā spēles fāzē!
	\end{enumerate}
}

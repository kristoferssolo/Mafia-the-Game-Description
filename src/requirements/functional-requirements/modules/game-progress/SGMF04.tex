\moduleFunctionTable
{Spēles stāvokļa detaļas}
{mod-func-progress-state-overview}
{Spēles stāvokļa detaļas}
{SGMF04}
{
	Funkcijas mērķis par spēles tagadējo stāvokli, kas ietver spēlētāju stāvokli, nakts numuru un atļautās darbības.
}
{
	Paslēptā informācija nosaka, vai izvade saturēs lomu datus par spēlētājiem.
	Obligātie parametri:
	Spēles istabas identifikators - vesels pozitīvs skaitlis.
	Paslēptā informācija - karodziņš.

}
{
	\begin{enumerate}
		\item Meklē spēles istabu attiecīgā tabulā, izmantojot darbības identifikatoru.
		      Ja spēles istaba netiek atrasta parāda 1. paziņojumu.
		\item Iesāk vārdnīcas gatavošanu ar noteiktu informāciju.
		\item Vārdnīcai pievieno spēles stāvokli un nakts numuru.
		\item Ja paslēptās informācijas karodziņš ir patiesība, katram spēlētājam sameklē datubāzē attiecīgo lomu.
		      Ja loma netika atrasta, parāda 2. paziņojumu ar attiecīgajiem spēlētāju identifikatoriem.
		      Pievieno vārdnīcai spēlētāju vārdnīcu sarakstu, katram pievienojot lomas identifikatoru, dzīvības stāvokli un sakaru redzamības mainīgos.
	\end{enumerate}
}
{
	Izvades mērķis ir nepieciešamās informācijas apkopošana vārdnīcā.
	Ja tiek izvadīta paslēptā informācija, vārdnīca sastāvēs no atribūtiem:
	\begin{enumerate}
		\item Spēles stāvoklis - skaitlisks kods.
		\item Nakts numurs - vesels pozitīvs skaitlis.
		\item Spēlētāju vārdnīcu saraksts. Katra spēlētāja vārdnīca, kā atslēgas ir:
		      \begin{enumerate}
			      \item Lomas identifikators - veselu pozitīvu skaitli;
			      \item Dzīvības stāvoklis - karodziņš;
			      \item Mafijas sakaru redzamība - karodziņš;
			      \item Vispārīgo sakaru redzamība - karodziņš;
		      \end{enumerate}

	\end{enumerate}
	Ja paslēptā informācija izvadīta netiks, tad spēlētāju vārdnīca sastāvēs tikai no lomas identifikatora.
}
{
	\begin{enumerate}
		\item Spēles istaba ar identifikatoru [istabas identifikators] netika atrasta!
		\item Spēlētāju ar identifikatoru: [spēlētāju identifikatori] netika atrasti!
	\end{enumerate}
}

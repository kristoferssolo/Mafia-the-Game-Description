\moduleFunctionTable
{Spēles uzstādījumu pārskats}
{mod-func-setup-overview}
{Spēles uzstādījumu pārskats}
{SUMF01}
{
	Funkcijas mērķis ir sniegt pārskatu par spēles uzstādījumiem.
}
{
	Obligātie parametri:
	\begin{enumerate}
		\item Spēles uzstādījumu identifikators - pozitīvs skaitlis.
	\end{enumerate}
}
{
	\begin{enumerate}
		\item Pārbauda, vai lietotājs ir reģistrēts.
		      Ja nav, izvada 1. paziņojumu.
		\item Sagatavo datubāzes pieprasījumu no spēles uzstādījumu tabulas.
		\item Pieprasījumam pievieno atlasīšanu pēc spēles uzstādījuma identifikatora.
		\item Sagatavo pieprasījumu lauku sarakstu.
		      Saraksta pamatā ir nosaukums, apraksts, vai ir pamata, vai ir aktīvs, izveidošanas laiks.
		\item Veic sagatavoto pieprasījumu, pieprasot iepriekš sagatavoto lauku sarakstu.
		      Ja pieprasījums neizdodas, parāda 2. paziņojumu.
		      Ja spēles uzstādījums netika atrasts, parāda 3. paziņojumu.
	\end{enumerate}
}
{
	Izvades datu mērķis ir spēles virtuālās istabas uzstādījumu datu izvadīšana.
	\begin{enumerate}
		\item Vārdnīca - nosaukums - simbolu virkne, apraksts - simbolu virkne, ir pamata - karodziņš, ir aktīvs - karodziņš, izveidošanas laiks - datums formatēts kā simbolu virkne
	\end{enumerate}
}
{
	\begin{enumerate}
		\item Darbība nav autorizēta!
		\item Notika sistēmas iekšējā kļūda! Mēģiniet vēlreiz!
		\item Tāds spēles uzstādījums nav atrasts! Mēģiniet vēlreiz!
	\end{enumerate}
}

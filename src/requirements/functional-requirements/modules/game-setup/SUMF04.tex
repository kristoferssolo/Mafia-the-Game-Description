\moduleFunctionTable
{Spēles uzstādījuma rediģēšana}
{mod-func-setup-edit}
{Spēles uzstādījuma rediģēšana}
{SUMF04}
{
	Funkcijas mērķis ir rediģēt spēles virtuālās istabas uzstādījumus.
}
{
	Ievades dati tiek saņemti no maksas lietotājiem un administratoriem pieejamās veidlapas.

	Obligātie parametri:
	\begin{enumerate}
		\item Spēles uzstādījumu identifikators - pozitīvs skaitlis.
		      Noklusētā vērtība - no konteksta spēles uzstādījumu, kas tiek rediģēts, iegūtais identifikators.
		\item Nosaukums - simbolu virkne ar garumu līdz 128 simboliem, kas var saturēt burtciparu simbolus, skaitļus, defises, pasvītras, apostrofus.
		\item Apraksts - simbolu virkne ar garumu līdz 2048 simboliem.
	\end{enumerate}

	Administratoram specifiskie ievaddati:
	\begin{enumerate}
		\item Vai ir pamata - karodziņš, noklusētā vērtība - nepatiess.
		\item Izveidošanas laiks - datums formatēts kā simbolu virkne, noklusētā vērtība - tagadējais laiks.
	\end{enumerate}
}
{
	\begin{enumerate}
		\item Ja lietotājs nav administrators vai lietotāja identifikators, nesakrīt ar nesakrīt ar lietotāja identifikatoru, kurš izveidoja doto spēles uzstādījumu, parādīt 6. paziņojumu.
		\item Veido izmainīto datu sarakstu pēc turpmāk izmainītajiem laukiem.
		\item Pārbauda, vai visi obligātie lauki ir iesniegti.
		      Ja nav, iegūst sarakstu ar neaizpildītajiem laukiem un parāda 1. paziņojumu.
		\item Pārbauda, vai nosaukums un darbības nosaukums, ja ievadīts, satur tikai pieļaujamos simbolus.
		      Ja nesatur, tad iegūst izmantotos neatļautos simbolus, tad parāda 9. paziņojumu ar attiecīgi laukiem un simboliem.
		\item Pārbauda, vai nosaukums un apraksts nepārsniedz noteikto garumu.
		      Ja pārsniedz, tad iegūst pārsniegto garumu parametru sarakstu un parāda 2. paziņojumu ar attiecīgajiem laukiem un garumiem.
		\item Ja tika iesniegts atšķirīgs nosaukums, mēģina sameklēt datubāzē uzstādījumus ar ievadīto nosaukumu.
		      Ja tāds pastāv, tad parāda 3. paziņojumu.
		\item Iepriekš izmainītos laukus pievieno izmainīto lauku sarakstam.
		\item Ja lietotājs ir administrators, pārbauda, vai datumam ir korekts formāts.
		      Ja nav, parāda 7. paziņojumu. Pārbauda, vai datums ir pagātnē vai tagad.
		      Ja datums ir nākotnē, parāda 8. paziņojumu.
		      Sagatavotiem datiem pievieno administratoriem specifiskās.
		\item Spēles uzstādījumu sagatavotie dati - lauki, kas ir rediģēto lauku sarakstā, tiek ierakstīti datubāzē.
		      Ja ierakstīšana nenotiek, parādīt 5. paziņojumu.
	\end{enumerate}
}
{
	Izvades datu mērķis ir noteikt spēles uzstādījumu rediģēšanas stāvokli.
	\begin{enumerate}
		\item Spēles uzstādījumu rediģēšanas apstiprinājuma stāvoklis - kods ar noteiktu stāvokli.
	\end{enumerate}
}
{
	\begin{enumerate}
		\item Lauks: [neaizpildīto lauku saraksts] netika aizpildīts (/-i)!
		\item {}[Parametra nosaukums] nedrīkst pārsniegt [noteikto parametra maksimālo simbolu skaits]!
		\item Uzstādījumi ar tādu nosaukumu jau eksistē! Samainiet nosaukumu un mēģiniet vēlreiz!
		\item Uzstādījumi veiksmīgi saglabāti!
		\item Spēles uzstādījumu rediģēšana nav veiksmīga!
		\item Darbība nav autorizēta!
		\item Nekorekts datums! Datuma formāts: [nepieciešamais datuma formāts].
		\item Izveidošanas datums nedrīkst būt nākotnē!
		\item {}[Parametra nosaukums] nedrīkst saturēt: [izmantoto parametra neatļauto simbolu saraksts]!
	\end{enumerate}
}

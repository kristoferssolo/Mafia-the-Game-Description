\moduleFunctionTable
{Jauna spēles uzstādījuma izveidošana}
{mod-func-setup-new}
{Jauna spēles uzstādījuma izveidošana}
{SUMF03}
{
	Funkcijas mērķis ir izveidot spēles virtuālās istabas uzstādījumus, turpmākai izmantošanai spēlē.
}
{
	Ievades dati tiek saņemti no maksas lietotājiem un administratoriem pieejamās veidlapas.

	Obligātie parametri:
	\begin{enumerate}
		\item Nosaukums - simbolu virkne ar garumu līdz 128 simboliem, kas var saturēt burtciparu simbolus, skaitļus, defises, pasvītras, apostrofus.
		\item Apraksts - simbolu virkne ar garumu līdz 2048 simboliem.
	\end{enumerate}
}
{
	\begin{enumerate}
		\item Pārbauda, vai visi obligātie lauki ir iesniegti.
		      Ja nav, iegūst sarakstu ar neaizpildītajiem laukiem un parāda 1. paziņojumu.
		\item Pārbauda, vai nosaukums un darbības nosaukums, ja ievadīts, satur tikai pieļaujamos simbolus.
		      Ja nesatur, tad iegūst izmantotos neatļautos simbolus, tad parāda 6. paziņojumu ar attiecīgi laukiem un simboliem.
		\item Pārbauda, vai nosaukums un apraksts nepārsniedz noteikto garumu.
		      Ja pārsniedz, tad iegūst pārsniegto garumu parametru sarakstu un parāda 2. paziņojumu ar attiecīgajiem laukiem un garumiem.
		\item Mēģina sameklēt datubāzē uzstādījumus ar ievadīto nosaukumu.
		      Ja tāds pastāv, tad parāda 3. paziņojumu.
		\item Jaunas spēles uzstādījumi dati tiek ierakstīti datubāzē.
		      Ja ierakstīšana notiek, parādīt 4. paziņojumu, citādi parādīt 5. paziņojumu.
	\end{enumerate}
}
{
	Izvades datu mērķis ir noteikt, vai spēles uzstādījumi ir veiksmīgi saglabāti.
	\begin{enumerate}
		\item Uzstādījumu izveidošanas stāvoklis - kods ar noteiktu stāvokli.
	\end{enumerate}
}
{
	\begin{enumerate}
		\item Lauks: [neaizpildīto lauku saraksts] netika aizpildīts (/-i)!
		\item {}[Parametra nosaukums] nedrīkst pārsniegt [noteikto parametra maksimālo simbolu skaits]!
		\item Uzstādījumi ar tādu nosaukumu jau eksistē! Samainiet nosaukumu un mēģiniet vēlreiz!
		\item Uzstādījumi veiksmīgi saglabāti!
		\item Sistēmas iekšēja kļūda! Mēģiniet vēlreiz!
		\item {}[Parametra nosaukums] nedrīkst saturēt: [izmantoto parametra neatļauto simbolu saraksts]!
	\end{enumerate}
}

\moduleFunctionTable
{Abonementu pārskats}
{mod-func-premium-overview}
{Abonementu pārskats}
{MAMF02}
{
	Funkcijas mērķis ir lietotājiem rādīt informāciju par maskas esošiem un bijušiem maksas abonementiem. Šī pārskata dati tiek kārtoti un filtrēti neobligātā kārtā.
}
{
	Ievades dati tiek saņemti no vietrāža parametriem.
	Neobligātie parametri:
	Lappuses numurs - pozitīvs skaitlis.
	Stāvokļa filtra opcija  (kods) - skaitlis, kas pieņem vairākas vērtības: 0 - filtru nepielietot, pārējie iespējamie abonementa stāvokļi.
	Kārtošanas opcija (kods) - skaitlis, kas pieņem 2 vērtības: 0 - rezultātus nekārtot, 1 -  rezultātus kārtot augoši.
}
{
	\begin{enumerate}
		\item Sagatavo datubāzes pieprasījumu no maksas abonementiem.
		\item Ja pastāv stāvokļa filtra opcija, kā vērtība nav nulle, pievieno pieprasījumam nosacījumu atlasīšanai pēc stāvokļa.
		\item Ja pastāv kārtošanas opcija un tā nav 0, pieprasījumam pievieno rezultātu kārtošanas nosacījumu.
		\item Iegūst lietotāja lomu sistēmā.
		      Ja lietotājs nav administrators, pieprasījumam pievieno nosacījumu, ka abonementa ierakstiem jābūt saistītiem ar lietotāju.
		\item Pieprasa ierakstu saskaitīšanu, izmantojot sagatavoto pieprasījumu.
		      Ja rezultātu skaits ir 0, tad parāda 1. paziņojumu.
		\item Aprēķina lappušu skaitu ar formulu: $L = ceil(Q / Q_l)$, $Q$ - rezultātu skaits, $Q_l$ - ierakstu skaits vienā lappusē.
		\item Ja ievaddatos nav lappuses numurs vai tā pārsniedz aprēķināto lappušu skaitu, tad turpmāk lappuses numurs būs 1.
		\item Aprēķina ierakstu nobīdi ar formulu: $O = (N - 1) * Q_l$, kur $O$ - nobīde; $N$ - lappuses numurs, $Q_l$ - ierakstu skaits vienā lappusē.
		\item Pievieno pieprasījumam nobīdi pēc aprēķinātās lappuses.
		\item Veic sagatavoto pieprasījumu, iegūstot abonementa sākuma laiku, beigu laiku, stāvokli, nākamā maksājuma laiku, maksājumu atstrādāja klienta identifikators.
		      Ja pieprasījums neizdodas, parāda 2. paziņojumu.
	\end{enumerate}
}
{
	Izvades datu mērķis ir parādīt pārskata, ņemot vērā filtrus un kārtošanu, ja tas tika pieprasīts.
	Kā arī dot informāciju abonementa abonementa atcelšanai.

	Izvades dati:
	\begin{enumerate}
		\item Pārskata ieraksta dati: vairāki ieraksti, kas sastāv no 2 datumiem - sākuma un beigu laiks (kas varētu neeksistēt),
		      abonementa stāvoklis - skaitlisks kods, kas apzīmē tekošo stāvokli abonementam, maksājumu atstrādāja klienta identifikators - skaitlis.
		\item Lapu skaits - pozitīvs skaitlis.
		\item Tekošā lappuse - pozitīvs skaitlis, kas ir mazāks vai vienāds par lapu skaitu.
	\end{enumerate}
}
{
	\begin{enumerate}
		\item Netika atrasts neviens abonements!
		\item Notika sistēmas iekšējā kļūda! Mēģiniet vēlreiz vēlāk!
	\end{enumerate}
}

\moduleFunctionTable
{Lomas rediģēšana}
{mod-func-role-edit}
{Lomas rediģēšana}
{SLMF05}
{
	Funkcijas mērķis ir rediģēt spēles lomas.
}
{
	Ievades dati tiek saņemti no maksas lietotājiem un administratoriem pieejamās veidlapas.

	Obligātie parametri:
	\begin{enumerate}
		\item Spēles lomas identifikators - pozitīvs skaitlis. Noklusētā vērtība - no konteksta spēles loma, kas tiek rediģēts, iegūtais identifikators.
		\item Nosaukums - simbolu virkne ar garumu līdz 64 simboliem, kas var saturēt burtciparu simbolus, skaitļus, defises, pasvītras, apostrofus.
		\item Apraksts - simbolu virkne ar garumu līdz 2048 simboliem.
	\end{enumerate}

	Neobligātie parametri:
	\begin{enumerate}
		\item Maksimāli pieļaujamais skaits spēlē - nenegatīvs skaitlis.
		\item Vai var tikt mafijas noslepkavots - karodziņš.
		\item Lomas darbības nosaukums - virkne -  simbolu virkne ar garumu līdz 64 simboliem, kas var saturēt burtciparu simbolus, skaitļus, defises.
		\item Vai lomas darbība ir tūlītēja - virkne - karodziņš.
	\end{enumerate}

	Administratoram specifiskie ievaddati:
	\begin{enumerate}
		\item Vai ir lietotāju izveidota - karodziņš.
	\end{enumerate}
}
{
	\begin{enumerate}
		\item Ja lietotājs nav administrators vai lietotāja identifikators, nesakrīt ar nesakrīt ar lietotāja identifikatoru, kurš izveidoja doto spēles lomu, parādīt 6. paziņojumu.
		\item Veido izmainīto datu sarakstu pēc turpmāk izmainītajiem laukiem.
		\item Pārbauda, vai nosaukums un darbības nosaukums, ja ievadīts, satur tikai pieļaujamos simbolus.
		      Ja nesatur, tad iegūst izmantotos neatļautos simbolus, tad parāda 8. paziņojumu ar attiecīgi laukiem un simboliem.
		\item Pārbauda, vai visi obligātie lauki ir iesniegti.
		      Ja nav, iegūst sarakstu ar neaizpildītajiem laukiem un parāda 1. paziņojumu.
		\item Pārbauda, vai nosaukums un apraksts nepārsniedz noteikto garumu.
		      Ja pārsniedz, tad iegūst pārsniegto garumu parametru sarakstu un parāda 2. paziņojumu ar attiecīgajiem laukiem un garumiem.
		\item Pārbauda, vai maksimāli pieļaujamais skaits spēlē ir nenegatīvs skaitlis.
		      Ja nav, tad parāda 7. paziņojumu.
		\item Ja tika iesniegts atšķirīgs nosaukums, mēģina sameklēt datubāzē lomu ar ievadīto nosaukumu.
		      Ja tāda pastāv, tad parāda 3. paziņojumu.
		\item Iepriekš izmainītos laukus pievieno izmainīto lauku sarakstam.
		\item Spēles uzstādījumu sagatavotie dati - lauki, kas ir rediģēto lauku sarakstā, tiek ierakstīti datubāzē.
		      Ja ierakstīšana nenotiek, parādīt 5. paziņojumu.
	\end{enumerate}
}
{
	Izvades datu mērķis ir noteikt, vai spēles loma ir veiksmīgi rediģēta.
	\begin{enumerate}
		\item Lomas rediģēšanas stāvoklis - kods ar noteiktu stāvokli.
	\end{enumerate}
}
{
	\begin{enumerate}
		\item Lauks: [neaizpildīto lauku saraksts] netika aizpildīts (/-i)!
		\item {}[Parametra nosaukums] nedrīkst pārsniegt [noteikto parametra maksimālo simbolu skaits]!
		\item Loma ar tādu nosaukumu jau eksistē! Samainiet nosaukumu un mēģiniet vēlreiz!
		\item Loma veiksmīgi saglabāti!
		\item Lomas rediģēšana nav veiksmīga!
		\item Darbība nav autorizēta!
		\item Skaitlim jābūt nenegatīvam! Mēģiniet vēlreiz!
		\item {}[Parametra nosaukums] nedrīkst saturēt: [izmantoto parametra neatļauto simbolu saraksts]!
	\end{enumerate}
}

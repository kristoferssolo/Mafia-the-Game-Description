\moduleFunctionTable
{Jaunas lomas izveidošana}
{mod-func-role-create}
{Jaunas lomas izveidošana}
{SLMF04}
{
	Funkcijas mērķis ir izveidot spēles lomu.
	Ievades dati tiek saņemti no maksas lietotājiem un administratoriem pieejamās veidlapas.

}
{
	Obligātie parametri:
	\begin{enumerate}
		\item Nosaukums - simbolu virkne ar garumu līdz 64 simboliem, kas var saturēt burtciparu simbolus, skaitļus, defises.
		\item Apraksts - simbolu virkne ar garumu līdz 2048 simboliem.
		\item Neobligātie parametri:
		\item Maksimāli pieļaujamais skaits spēlē - nenegatīvs skaitlis, noklusētā vērtība - 1.
		\item Vai var tikt mafijas noslepkavots - karodziņš, noklusētā vērtība - patiess.
		\item Lomas darbības nosaukums - virkne -  simbolu virkne ar garumu līdz 64 simboliem, kas var saturēt burtciparu simbolus, skaitļus, defises.
		\item Vai lomas darbība ir tūlītēja - virkne -  karodziņš, noklusētā vērtība - nepatiess.
	\end{enumerate}
	Administratoram specifiskie ievaddati:
	\begin{enumerate}
		\item Vai ir lietotāju izveidota - karodziņš, noklusētā vērtība - patiess.
	\end{enumerate}
}
{
	\begin{enumerate}
		\item Pārbauda, vai visi obligātie lauki ir iesniegti.
		      Ja nav, iegūst sarakstu ar neaizpildītajiem laukiem un parāda 1. paziņojumu.
		\item Pārbauda, vai nosaukums un darbības nosaukums, ja ievadīts, satur tikai pieļaujamos simbolus.
		      Ja nesatur, tad iegūst izmantotos neatļautos simbolus, tad parāda 7. paziņojumu ar attiecīgi laukiem un simboliem.
		\item Pārbauda, vai nosaukums, apraksts un darbības nosaukums, ja ievadīts, nepārsniedz noteikto garumu.
		      Ja pārsniedz, tad iegūst pārsniegto garumu parametru sarakstu un parāda 2. paziņojumu ar attiecīgajiem laukiem un garumiem.
		\item Ja ir ievadīts maksimāli pieļaujamais skaits spēlē parametrs, pārbauda, vai skaitlis ir nenegatīvs, citādi parāda 6. paziņojumu.
		\item Ja ir ievadīts(/-i) lomas darbības nosaukums(/-i), tad izveidot daudz pret daudz attiecību tabulu starp spēles lomas un lomas darbības tabulām.
		\item Ja lietotājs nav administrators, iestatīt noklusēto vērtību ``vai ir lietotāju izveidota'' uz patiess.
		\item Mēģina sameklēt datubāzē spēles lomu ar ievadīto nosaukumu.
		      Ja tāds pastāv, tad parāda 3. paziņojumu.
		\item Jaunas spēles lomas dati tiek ierakstīti datubāzē.
		      Ja ierakstīšana notiek, parādīt 4. paziņojumu, citādi parādīt 5. paziņojumu.
	\end{enumerate}
}
{
	Izvades datu mērķis ir noteikt, vai spēles loma ir veiksmīgi saglabāta.
	\begin{enumerate}
		\item Lomas izveidošanas stāvoklis - kods ar noteiktu stāvokli.
	\end{enumerate}
}
{
	\begin{enumerate}
		\item Lauks: [neaizpildīto lauku saraksts] netika aizpildīts (/-i)!
		\item {}[Parametra nosaukums] nedrīkst pārsniegt [noteikto parametra maksimālo simbolu skaits]!
		\item Uzstādījumi ar tādu nosaukumu jau eksistē! Samainiet nosaukumu un mēģiniet vēlreiz!
		\item Uzstādījumi veiksmīgi saglabāti!
		\item Sistēmas iekšēja kļūda! Mēģiniet vēlreiz!
		\item Skaitlim jābūt nenegatīvam! Mēģiniet vēlreiz!
		\item {}[Parametra nosaukums] nedrīkst saturēt: [izmantoto parametra neatļauto simbolu saraksts]!
	\end{enumerate}
}

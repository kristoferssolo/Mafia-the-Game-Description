\moduleFunctionTable
{Lomas detaļu pārskats}
{mod-func-role-details}
{Lomas detaļu pārskats}
{SLMF01}
{
	Funkcijas mērķis ir izvadīt specificētās lomas pārskatu.
}
{
	Obligātie parametri:
	\begin{enumerate}
		\item Spēles lomas identifikators - pozitīvs skaitlis.
	\end{enumerate}
}
{
	\begin{enumerate}
		\item Pārbauda vai lietotājs ir reģistrēts.
		      Ja nav, izvada 1. paziņojumu.
		\item Sagatavo datubāzes pieprasījumu no spēles lomas tabulas.
		\item Pieprasījumam pievieno atlasīšanu pēc spēles lomas identifikatora.
		\item Sagatavo pieprasījuma lauku sarakstu.
		      Saraksta pamatā ir nosaukums, apraksts, maksimāli pieļaujamais skaits spēlē, vai var tikt mafijas noslepkavots, vai ir lietotāja izveidots.
		\item Veic sagatavoto pieprasījumu, pieprasot iepriekš sagatavoto lauku sarakstu.
		      Ja pieprasījums neizdotas, parāda 2. paziņojumu.
		      Ja spēles loma netika atrasta, parāda 3. paziņojumu.
	\end{enumerate}
}
{
	Izvades datu mērķis ir spēles lomas datu izvadīšana.
	\begin{enumerate}
		\item Vārdnīca - nosaukums - simbolu virkne, apraksts - simbolu virkne, maksimāli pieļaujamais
		      skaits spēlē - nenegatīvs skaitlis, vai var tikt mafijas noslepkavots - karodziņš, vai ir lietotāja izveidots - karodziņš.
	\end{enumerate}
}
{
	\begin{enumerate}
		\item Darbība nav autorizēta!
		\item Notika sistēmas iekšēja kļūda! Mēģiniet vēlreiz!
		\item Tāda spēles lomas nav atrasta! Mēģiniet vēlreiz!
	\end{enumerate}
}

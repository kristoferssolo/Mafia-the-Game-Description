\moduleFunctionTable
{Lomas dzēšana}
{mod-func-role-delete}
{Lomas dzēšana}
{SLMF06}
{
	Funkcijas mērķis ir neatgriezeniski dzēst spēles lomu.
}
{
	Ievades dati tiek saņemti no maksas lietotājiem un administratoriem pieejamās veidlapas.
	Obligātie parametri:
	\begin{enumerate}
		\item Spēles lomas identifikators - pozitīvs skaitlis.
	\end{enumerate}
}
{
	\begin{enumerate}
		\item Ja lietotājs nav administrators vai lietotājs, kurš izveidoja doto lomu, parādīt 1. paziņojumu.
		\item Pārbauda, vai loma ar tādu identifikatoru eksistē.
		      Ja neeksistē, parāda 2. paziņojumu.
		\item Visas spēles lomas un lomas darbības daudz pret daudz starptabulas izdzēš.
		      Ja izdzēšāna nav veiksmīga, parāda 4. paziņojumu.
		\item Spēles lomas ierakstu izdzēš.
		      Ja izdzēšana ir veiksmīga, parāda 3. paziņojumu.
		      Ja izdzēšāna nav veiksmīga, parāda 4. paziņojumu.
	\end{enumerate}
}
{
	Izvades datu mērķis ir noteikt spēles lomas dzēšanas stāvokli.
	\begin{enumerate}
		\item Spēles lomas dzēšanas stāvoklis - kods ar noteiktu stāvokli.
	\end{enumerate}
}
{
	\begin{enumerate}
		\item Darbība nav autorizēta!
		\item Tāda spēles loma nav atrasta! Mēģiniet vēlreiz!
		\item Spēles loma ir veiksmīgi izdzēsta!
		\item Sistēmas iekšēja kļūda! Mēģiniet vēlreiz!
	\end{enumerate}
}

\procedureTable
{
	Pārskata pieprasījuma sagatavošana
}
{KPR07}
{
	Sagatavo entitātes pārskata datubāzes pieprasījumu pievienojot neobligātu lappuses nobīdi, filtrēšanu un kārtošanu.
}
{
	Obligātie parametri:
	\begin{enumerate}
		\item Datubāzes pieprasījums, kas atbilst \hyperref[tab:IIDP13]{IIDP13}.
	\end{enumerate}

	Neobligātie parametri:
	\begin{enumerate}
		\item Lappuses numurs - vesels pozitīvs skaitlis;
		\item Meklēšanas uzvedne - simbolu virkne ar garumu līdz 50 simboliem bez atļauto simbolu ierobežojumiem;
		\item Kārtošanas vārdnīcu saraksts, kas sastāv no vārdnīcām: datu bāzes atribūta nosaukums (atbilst \hyperref[tab:IDP12]{IDP12}) - kārtošanas kods (atbilst \hyperref[tab:IDP11]{IDP11});
		\item Filtru vārdnīcu saraksts, kas sastāv no vārdnīcām: datu bāzes atribūta nosaukums (atbilst \hyperref[tab:IDP12]{IDP12}) - filtra vērtība (vesels skaitlis) un filtra veids (0 - Būla mainīgā filtrs, 1 - entitātes identifikatora filtrs).
	\end{enumerate}
}
{
	\begin{enumerate}
		\item Ja filtru vārdnīcu saraksts nav tukšs, katram saraksta elementam pievieno kārtošanu pieprasījumam ar attiecīgiem atribūtu nosaukumiem, filtra veidiem un vērtībām, izmantojot \hyperref[tab:KPR04]{KPR04};
		\item Ja meklēšanas uzvedne ir iesniegta un nav tukša simbolu virkne, tad pieprasījumam pievieno meklēšanas nosacījumu meklēšanai pēc pilna vārda, segvārda un biogrāfijas, izmantojot \hyperref[tab:KPR05]{KPR05};
		\item Ja kārtošanas vārdnīcu saraksts nav tukšs, katram saraksta elementam pievieno kārtošanu pieprasījumam ar attiecīgiem atribūtu nosaukumiem, kārtošanas kodiem, izmantojot \hyperref[tab:KPR06]{KPR06};
		\item Ja lappuses numurs netika iesniegts, uzskata, ka lappuses numurs ir 1;
		\item Pieprasa ierakstu saskaitīšanu, izmantojot sagatavoto pieprasījumu.
		      \begin{enumerate}
			      \item Ja rezultātu skaits ir lielāks par 0, iegūst lappuses ierakstu nobīdi, lappuses numuru un kopējo lappušu skaitu, izmantojot \hyperref[tab:KPR01]{KPR01} ar attiecīgo rezultātu skaitu, lappuses numuru, noklusēto ierakstu skaitu lappusē;
			      \item Ja rezultātu skaits ir 0, kopējo lappušu skaitu un lappuses numuru uzskata par 0.
		      \end{enumerate}
	\end{enumerate}
}
{
	\begin{enumerate}
		\item Datubāzes pieprasījums, kas atbilst \hyperref[tab:IIDP13]{IIDP13};
		\item Lappuses numurs - vesels pozitīvs skaitlis;
		\item Kopējs lappušu skaits - vesels pozitīvs skaitlis.
	\end{enumerate}
}

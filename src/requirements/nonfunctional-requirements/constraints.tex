\subsubsection{Projekta ierobežojumi}

\paragraph{Intelektuālā īpašuma tiesības}

    Projektā jāievēro autortiesību un preču zīmju likumi. Lai izvairītos no
    juridiskām problēmām, ir jāsaņem atbilstošas atļaujas un licences par spēļu
    aktīviem (assets).

\paragraph{Atbilstība standartiem}

    Šajā sadaļā ir izklāstīti galvenie standarti, kā prasībām sistēmai ir
    jāatbilst, lai nodrošinātu robustumu, uzticamību un lietotāju tiesību un
    cerību ievērošanu.

    \subparagraph{Datu privātuma atbilstība}

        Sistēmai ir jāatbilst datu aizsardzības noteikumiem, tostarp
        vispārīgajai datu aizsardzības regulai (GDPR - General Data Protection
        Regulation). Ir būtiski noteikt visaptverošu datu apstrādes praksi, lai
        nodrošinātu lietotāju datu konfidencialitāti un drošību.

    \subparagraph{Pieejamības standarti}

        Lai nodrošinātu vienlīdzīgu piekļuvi visiem lietotājiem, sistēmai
        jāatbilst tīmekļa satura pieejamības vadlīnijām (WCAG - Web Content
        Accessibility Guidelines) attiecībā uz tīmekļa pieejamību. Jāapsver
        lokalizācijas un iekļaušanas iespējas, lai pielāgotos dažādām
        auditorijām.

    \subparagraph{Drošības standarti}

        Jāievieš stingri drošības pasākumi, ievērojot nozares paraugpraksi,
        piemēram, atvērto lietojumprogrammu drošības projektu visā pasaulē
        (OWASP - Open Worldwide Application Security Project), lai pasargātu no
        bieži sastopamām tīmekļa ievainojamībām. Datu aizsardzībai jāizmanto
        šifrēšanas protokoli.

\paragraph{Aparatūras ierobežojumi}

    \subparagraph{Atbalstītās ierīces}

        Sistēmai jābūt saderīgai ar dažādām ierīcēm, tostarp galddatoriem,
        klēpjdatoriem, viedtālruņiem un planšetdatoriem. Lai nodrošinātu
        netraucētu spēlēšanu, jānosaka minimālās aparatūras prasības.

    \subparagraph{Serveris un mitināšana}

        Uz servera, kurāuz kā tiek mitināta sistēma, jābūt pieejamam docker
        programmatūras atbalstam, lai standartizētu programmatūras izvietošanu
        ražošanas vidē un ražošanas vides simulēšanai un atvieglotai
        tehnoloģiju versiju sinhronizācijai ar izstrādes vidi. 

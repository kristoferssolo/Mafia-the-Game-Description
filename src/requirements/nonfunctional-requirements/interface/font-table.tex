

\begin{tabularx}{\linewidth}{|X|X|X|X|X|}
	\caption{Teksta elementu noformējuma specifikācija} \label{tab:font-size}                                                                                                                            \\

	\hline
	\textbf{Teksta elementa klase}             & \textbf{Izmērs} & \textbf{Svars}              & \textbf{Stils}       & \textbf{Cits}                                                                    \\ \hline
	\endfirsthead

	\hline \multicolumn{5}{r}{Turpinājums no iepriekšējās lapas}                                                                                                                                         \\ \hline
	\textbf{Teksta elementa klase}             & \textbf{Izmērs} & \textbf{Svars}              & \textbf{Stils}       & \textbf{Cits}                                                                    \\ \hline
	\endhead

	\hline \multicolumn{5}{r}{Turpinājums nākamajā lapā}                                                                                                                                                 \\ \hline
	\endfoot

	\hline
	\endlastfoot

	H1: Galvenais virsraksts                   & vismaz 36px     & Bold (treknraksts)          & Regulārs             & Var izmantot atšķirīgu krāsu, lai izceltu no citiem virsrakstiem.                \\ \hline
	H2: Apakšvirsraksti                        & vismaz 24px     & Bold (treknraksts)          & Regulārs             & Var izmantot atšķirīgu krāsu, lai izceltu no pamata teksta.                      \\ \hline
	H3: Sekundārie virsraksti                  & 20px            & Semi-Bold (pus-treknraksts) & Regulārs             & Var izmantot atšķirīgu krāsu.                                                    \\ \hline
	H4: Apakšvirsraksti                        & 18px            & Medium (vidējs treknraksts) & Regulārs             & Izcelt ar atšķirīgu krāsu.                                                       \\ \hline
	H5                                         & 16px            & Regulārs (parasts)          & Regulārs             &                                                                                  \\ \hline
	H5                                         & 16px            & Regulārs (parasts)          & Regulārs             &                                                                                  \\ \hline
	H6                                         & 14px            & Regulārs (parasts)          & Regulārs             &                                                                                  \\ \hline
	Pamatteksts                                & 16px            & Regulārs (parasts)          & Regulārs             &                                                                                  \\ \hline
	Saturteksts                                & 14px            & Regulārs (parasts)          & Regulārs             & Var izmantot slīprakstu, lai akcentētu citātus vai īpaši teksta fragmentus.      \\ \hline
	Citāti un īpaši akcentēti teksti           & 16px            & Regulārs (parasts)          & Italics (slīpraksts) & Lai iezīmētu citātus vai īpaši akcentētu tekstus var izmantot slīprakstu.        \\ \hline
	Sarunas un komentāri                       & 14px            & Regulārs (parasts)          & Regulārs             &                                                                                  \\ \hline
	Standarta saites                           & 16px            & Regulārs (parasts)          & Regulārs             & Izmantot krāsas un pasvītrojumu, lai izceltu saites.                             \\ \hline
	Apmeklētās un neapmeklētās saites          & 16px            & Regulārs (parasts)          & Regulārs             & Izmantot krāsas un pasvītrojumu, lai atšķirtu apmeklētās un neapmeklētās saites. \\ \hline
	Numurētās un nenumurētās saraksta vienības & 16px            & Regulārs (parasts)          & Regulārs             & Izmantot atzīmes (bullets) vai numurēšanas stilu.                                \\ \hline
	Atzīmes (bullets)                          & 12px            & Regulārs (parasts)          & Regulārs             &                                                                                  \\ \hline
	Tabulu teksts un tabulu virsraksti         & 14px            & Regulārs (parasts)          & Regulārs             &                                                                                  \\ \hline
	Ievades lauki                              & 16px            & Regulārs (parasts)          & Regulārs             &                                                                                  \\ \hline
	Izvēles rūtiņas un radio pogas             & 12px            & Regulārs (parasts)          & Regulārs             & Var izmantot atšķirīgu krāsu.                                                    \\ \hline
	Ievades lauku kļūdu ziņojumi               &                 &                             &                      &                                                                                  \\ \hline
	Galvenās navigācijas saites                & 18px            & Bold (treknraksts)          & Regulārs             &                                                                                  \\ \hline
	Sānu paneļa navigācijas saites             & 14px            & Regulārs (parasts)          & Regulārs             &                                                                                  \\ \hline
\end{tabularx}

\paragraph{Drošība}

Sistēmas drošības atribūti ir kategorizēti sarakstā:
\begin{itemize}
	\item Autentifikācija:
	      \begin{itemize}
		      \item Paroles tiek glabātas šifrētā formātā, pirms šifrēšanas, tām pievieno sāli.
		      \item Sāls tiek saglabāts teksta formātā pie lietotāja ieraksta.
		      \item Paroles atiestatīšanai tiek izmantots marķieris ar derīguma termiņu, ģenerēts ar jaucējfunkciju.
	      \end{itemize}

	\item Autorizācija:
	      \begin{itemize}
		      \item Sistēmā ir jārealizē uz lomu bāzēta piekļuve, kas nosaka noteiktās darbības noteiktām sistēmas lietotāju grupām.
		      \item Autorizācijas pārbaude gan priekšgala saskarnē, gan aizmugursistēmā.
	      \end{itemize}

	\item Datu aizsardzība:
	      \begin{itemize}
		      \item Paroles, maksājuma informācijas un citu ierobežotas piekļuves datu šifrēšana, tos glabājot un sūtot.
		      \item Sistēmai nav jāglabā dati, kas nav nepieciešami sistēmas darbībai.
	      \end{itemize}

	\item Sesiju pārvalde:
	      \begin{itemize}
		      \item Izmanto tikai HTTP sīkdatnes.
		      \item Sesijas noildzes izmantošana.
		      \item Sesijas atpazīšana tiek realizēta, izmantojot sīkdatnes.
	      \end{itemize}

	\item Ievades pārbaude un dezinficēšana:
	      \begin{itemize}
		      \item Aizsardzība pret SQL injekcijām, komandu injekcijām un citiem injekciju veidiem.
		      \item Visu ārējo pieprasījumu dati ir validēti un sanitizēti, ja nepieciešams.
		      \item Ieviesiet pareizu validāciju visiem lietotāja ievadītajiem datiem un visiem datiem, kas saņemti no ārējām sistēmām.
	      \end{itemize}

	\item Aizsardzība pret tīmekļa apdraudējumiem:
	      \begin{itemize}
		      \item Aizsardzība pret CSRF) uzbrukumiem ar CSRF marķiera izmantošanu.
		      \item Aizsardzība pret starpvietņu-skriptošanas (XSS) uzbrukumiem.
	      \end{itemize}

	\item Kļūdu apstrāde:
	      \begin{itemize}
		      \item Kļūdu paziņojumi neatklāj ierobežotas ierobežotas piekļuves informāciju.
		      \item Kļūdu paziņojumi ir pielāgoti lietotāju grupai - sistēmas lietotāji, kas nepieder administratoru grupai, neredz ar sistēmas iekšējiem komponentiem saistīto informāciju, piemēram, datubāzes kļūmes.
	      \end{itemize}

	\item API:
	      \begin{itemize}
		      \item API marķieru izmantošana, izņemot publiski pieejamos API galapunktos.
		      \item API pieprasījumu ierobežojums laikā (vienam lietotājam un vienai IP adresei).
		      \item Visu API pieprasījumu ievades validācija un sanitizēšana, ja nepieciešams.
	      \end{itemize}

	\item Datņu augšuplādēšana:
	      \begin{itemize}
		      \item Datnes, kas tiek augšuplādētas, tiek analizētas pret ļaunatūru pirms uzglabāšanas.
		      \item Datņu lielums un datu tips tiek validēts.
	      \end{itemize}

	\item Rezerves kopēšana:
	      \begin{itemize}
		      \item Rezerves kopijas izveidošanai jānotiek noteiktā laika periodā automātiski lokālā glabātuvē.
		      \item Automatizēta procedūra sistēmas datu atjaunošanai, datu zaudēšanas vai bojājumu gadījumā.
	      \end{itemize}

	\item Darbību žurnāls:
	      \begin{itemize}
		      \item Darbību žurnāla uzturēšana.
		      \item Darbību žurnāls nesatur ierobežotas piekļuves informāciju.
		      \item Darbību žurnāla rezerves kopēšana atsevišķi no pamata rezerves kopijām.
	      \end{itemize}
\end{itemize}

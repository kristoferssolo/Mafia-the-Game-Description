\section*{Anotācija}
\setcounter{page}{2}
Sociālā lomu spēle ``Mafija'' ir plaši pazīstama. Tā ir pieejama vairākos
paveidos un formātos, kā arī piedāvā neierobežotu skaitu konfigurāciju un lomu.
Spēlētāju ērtībai tiek nodrošināts tīmekļa vietnes formāts, kas ietver
norādījumus, informējot lietotāju par pieejamajām iespējām, un skaidrojumus,
aprakstot spēles elementus un saskarni, ar mērķi vienkāršot tās spēlēšanu.
Tirgus izpēte apliecina, ka ir pieejami vairāki, nepilnvērtīgi risinājumi.
Programmatūras prasību specifikācija apraksta sistēmas pamatprasības ar
papildus funkcionalitāti, tostarp lomu klāsta papildināšanu, spēles
konfigurāciju izveidi, kā arī priviliģētu lietotāju (maksas lietotāju),
pielietojot abonementa paveida maksājumu sistēmu.

\textbf{Atslēgvārdi:}

Mafijas spēle, sistēmas prasības, specifikācijas dokuments, programmatūras
uzlabošana, lomu spēle, vienkāršota spēlēšana, organizatoriski risinājumi,
programmatūras prasību specifikācija, lietotāju veidots saturs, abonements,
maksas lietotājs, maksājumu apstrādātāja lietojumprogrammas saskarni (API).

\section*{Abstract}
The social role-playing game ``Mafia'' is widely known. It is available in
various versions and formats, offering an unlimited number of configurations
and roles. For the convenience of players, the game is made as a web
application that includes instructions, informing the user about available
options, and explanations describing the game's elements and interface, to make
it simpler. Market research confirms that several mediocre solutions are
available. The software requirements specification describes the system's basic
requirements with additional functionality, including expanding the range of
roles, creating game configurations, and a privileged user (premium user) using
a subscription-based payment system.

\textbf{Keywords:}

Mafia game, system requirements, specification document, software improvement,
role-playing, simplified gameplay, organizational solutions, software
requirements specification, user-generated content, subscription, premium user,
payment processor application program interface (API).
